
\section{Separación de backsheet}
\label{sec:Resultados backsheet}
Para preparar las muestras, se cortaron los paneles, con el marco de aluminio y caja de conexiones previamente desmantelados, en muestras de 10x10cm. Primero, el backsheet es separado mecánicamente. 

\subsection{Proceso mecánico para recuperar Backsheet } 
\label{resultados mecanicos backsheet}

Se utilizó una pistola de calor marca Stanley con controlador de temperatura y velocidad de aire para calentar la placa posterior (backsheet) del panel solar Para conocer los parámetros óptimos de separación, las velocidades (1: 300 L/min y 2: 500 L/min) y temperaturas de la pistola fueron evaluadas en ambos tipos de panel. 

La tabla \ref{tab:PistolaMono} muestra los resultados del panel monocristalino. La temperatura para que la placa se retire sin deteriorarse o tener residuos de EVA es de 100 a 130 con flujo de aire 1 (300 L/min), y de 80 a 90 °C en la placa, con velocidad 2 (500 L/min). Además la distancia entre la placa y la boca de la pistola debe ser entre 5 y 8 cm, lo ideal es no dejar más de 3 minutos calentando un solo lugar ya que después de este tiempo la temperatura de la placa puede superar los 140°C y empezar a deteriorarse. 
Los resultados obtenidos para las muestras M-inicial policristalinas (tabla \ref{tab:PistolaPoli}) fueron distintos. Se observó que, para lograr separar el backsheet de manera adecuada, la temperatura de la pistola debe mantenerse entre 60 y 70°C con una velocidad de aire en nivel 2. A temperaturas más altas, el EVA pierde estabilidad y comienza a derretirse, adquiriendo una consistencia similar a la del silicón frío. Por otro lado, si se utiliza una velocidad de aire menor, es necesario aplicar mayor fuerza con las pinzas, lo que con frecuencia provoca la ruptura de la muestra.

\begin{table}[htb]
	\caption{Parámetros de pistola de calor para la separación de backsheet monocristalino.}
	\vspace{-0.5em} %Para reducir el espacio después del caption
	\label{tab:PistolaMono}
	\begin{center}
		\begin{tabular}{|p{2cm}||p{3cm}|p{3cm}|p{4cm}|}\hline
			\textbf{Velocidad de salida de aire de pistola} & \textbf{Temperatura aire de la pistola} & \textbf{Temperatura de la placa}&\textbf{Observaciones}\\ \hline
			\textbf{1} & 39 °C & 37 °C & La temperatura no es suficiente para separar la placa \\ \hline
			\textbf{2} & 47 °C & 40 °C & Se necesita mantener el calor por al menos 3 minutos para que la placa empiece a despegarse. Se requiere aplicar mucha presión con las pinzas, lo que provoca que se rompa la placa y queden restos de EVA.\\ \hline
			\textbf{1} & 108 °C & 70 °C & La placa se despega aplicando fuerza con la pinza\\ \hline
			\textbf{2} & 150 °C & 105 °C & La placa se desprende fácilmente, sin embargo, después de un minuto aplicando calor en la misma zona empieza a deteriorarse.\\ \hline
			\textbf{1} & 190 °C & 135 °C & La placa se desprende fácilmente, sin embargo quedan residuos de EVA que comienza a hincharse y adherirse al backsheet\\ \hline	
			\textbf{2} & 260 °C & 148 °C & La placa emite vapor y se dobla, siendo difícil de manipular\\ \hline
		\end{tabular}
	\end{center}
\end{table} 
\clearpage

\begin{table}[htb]
	\caption{Parámetros de pistola de calor para la separación de backsheet policristalino.}
	\vspace{-0.5em} %Para reducir el espacio después del caption
	\label{tab:PistolaPoli}
	\begin{center}
		\begin{tabular}{|p{2cm}||p{3cm}|p{3cm}|p{4cm}|}\hline
			\textbf{Velocidad} & \textbf{Temperatura aire de la pistola} & \textbf{Temperatura de la placa}&\textbf{Observaciones}\\ \hline
			\textbf{1} & 39 °C & 37 °C & La temperatura no es suficiente para separar la placa \\ \hline
			\textbf{2} & 47 °C & 40 °C & Se necesita mantener el calor por al menos 4 minutos para que la placa empiece a despegarse. Se requiere aplicar mucha presión con las pinzas, lo que provoca que se rompa la placa y queden restos de EVA.\\ \hline
			\textbf{1} & 60°C & 49-52 °C & La placa no se desprende\\ \hline
			\textbf{2} & 74 °C & 60 °C & La placa se desprende aplicando fuerza con la pinza, si no se tiene cuidado la M-inicial se rompe.\\ \hline
			\textbf{1} & 90 °C & 83 °C & La placa se desprende fácilmente, sin embargo el EVA es menos estable térmicamente por lo que se deforma y desprende junto al backsheet como una pasta.\\ \hline	
			\textbf{2} & 100 & 96 °C & La placa emite vapor y se dobla, siendo difícil de manipular, el EVA comienza a derretirse.\\ \hline
		\end{tabular}
	\end{center}
\end{table} 

los resultados muestran que mientras en los monocristalinos se logra una separación limpia y controlada en un rango de 100 a 130°C con velocidad de aire baja, en los policristalinos la separación es más delicada, requiriendo temperaturas más bajas (60 a 70°C) y velocidad de aire alta para evitar la degradación térmica del EVA. Estas diferencias reflejan tanto la composición del encapsulante como la respuesta térmica de los distintos tipos de módulos, lo que implica que los procesos de reciclaje deben adaptarse específicamente al tipo de panel tratado.
\clearpage

\newpage
\section{Recuperación de vidrio} 

Para retirar el vidrio de la muestra obtenida del proceso anterior, conformada por vidrio, celda y encapsulante, se evalúan dos distintos métodos: químico y mecánico, descritos en la sección \ref{sec:Vidrio}. 

\subsection{Proceso químico para recuperar vidrio}

Para el método químico, 8 g de muestra fueron colocados en recipientes cerrados con solvente. Al finalizar se llevaron a una parrilla de agitación durante 5 min. La separación del vidrio fue mejor en las muestras de panel policristalino utilizando xileno y tolueno como solvente, así se observa en la figura \ref{fig:VidrioPoli}. Para el panel monocristalino, el vidrio no se separó con ciclohexano, luego de agitar la celda con EVA precipitó junto con el vidrio debido a que seguía adherido a este. Utilizando xileno como solvente algunos trozos de EVA con celda lograron separarse del vidrio, flotando en el recipiente, sin embargo parte de este material no se separó completamente, precipitando con el vidrio (figura \ref{fig:VidrioAgitado}). 

\begin{figure}[htb]
	\begin{center}
		\includegraphics[width=0.9\textwidth]{./Figuras/VidrioPoli.JPG}
	\end{center}
	\vspace{-1em} %Para reducir el espacio antes del caption
	\caption{Vidrio recuperado después del proceso químico y agitación de las muestras policristalina.}
	\label{fig:VidrioPoli}
\end{figure}

En los paneles policristalinos, el uso de xileno y tolueno permitió una separación más eficiente del vidrio, facilitando su recuperación con un menor grado de contaminación por EVA. En las muestras monocristalinas, el uso de ciclohexano no resultó efectivo, y aunque con xileno se logró separar parcialmente el EVA, una fracción importante permaneció adherida al vidrio. Estos resultados sugieren que la efectividad del proceso químico depende tanto del tipo de solvente como de la naturaleza del panel, lo que implica la necesidad de optimizar los parámetros en función de cada configuración de celda para mejorar la eficiencia del reciclaje.

\begin{figure}[htb]
	\begin{center}
		\includegraphics[width=0.6\textwidth]{./Figuras/VidrioAgitado.JPG}
	\end{center}
	\vspace{-1em} %Para reducir el espacio antes del caption
	\caption{Muestras al finalizar la agitación.}
	\label{fig:VidrioAgitado}
\end{figure}

\begin{table}[htb]
	\centering
	\caption{Pesos de EVA separado al triturar y tamizar.}
	\vspace{-0.5em} %Para reducir el espacio después del caption
	\label{tab:PesosvidrioR1}
	\begin{center}
		\begin{tabular}{|>{\centering}p{3cm}||>{\centering}p{3cm}|c|} \hline
			\textbf{Muestra} & \textbf{Peso inicial (g)} & \textbf{ vidrio (g)} \\ \hline
			\textbf{M1} Monocristal & 5 & 4.95 \\ \hline
			\textbf{M2} Monocristal & 5 & 4.48 \\ \hline
			\textbf{M3} Monocristal & 5 & 4.50 \\ \hline
			\textbf{M4} Monocristal & 5 & 4.57 \\ \hline
			\textbf{M5} Monocristal & 5 & 4.38 \\ \hline
			\textbf{M6} Monocristal & 5 & 4.59 \\ \hline
			\textbf{P1} Policristal & 5 & 3.85 \\ \hline
			\textbf{P2} Policristal & 5 & 4.29 \\ \hline
			\textbf{P3} Policristal & 5 & 4.23 \\ \hline
			\textbf{P4} Policristal & 5 & 4.26 \\ \hline
			\textbf{P5} Policristal & 5 & 4.15 \\ \hline
			\textbf{P6} Policristal & 5 & 4.19 \\ \hline
		\end{tabular}
	\end{center}
\end{table}

\clearpage

\subsection{Proceso mecánico para recuperación del vidrio} 
\label{subsec:vidrio mecánico resultados}

Para separar el vidrio mecánicamente, se utilizó una pistola de calor, manteniendo la temperatura entre 50 y 70 °C con velocidad de aire baja. Se colocó la muestra 5 cm de la pistola y se calentó durante 3 minutos. Una cuchilla se pasa entre el EVA y vidrio, desprendiéndolo de la muestra. Los parámetros de temperaturas utilizadas en este proceso se obtuvieron del análisis visto en el método mecánico para separar backsheet en la sección \ref{sec:Resultados backsheet}

\begin{figure}[htb]
	\begin{center}
		\includegraphics[width=0.6\textwidth]{./Figuras/pistolaVidrio.png}
	\end{center}
	\vspace{-1em} %Para reducir el espacio antes del caption
	\caption{Proceso mecánico para separar vidrio, muestra monocristalina}
	\label{fig:PistolaVidrio}
\end{figure} 
\clearpage
\newpage

Los pesos obtenidos de los componentes se presentan en la siguiente tabla 

\begin{table}[htb]
	\centering
	\caption{Pesos de materiales obtenidos después del proceso mecánico para separar vidrio.}
	\vspace{-0.5em} %Para reducir el espacio después del caption
	\label{tab:PesosVidrioR2}
	\begin{center}
		\begin{tabular}{|>{\centering\arraybackslash}p{3cm}||>{\centering\arraybackslash}p{3cm}|>{\centering\arraybackslash}p{3cm}|>{\centering\arraybackslash}p{2cm}|>{\centering\arraybackslash}p{2cm}|} \hline
			\textbf{Muestra} & \textbf{Peso total (g)} & \textbf{Backsheet (g)} & \textbf{Vidrio (g)} & \textbf{Celda + EVA (g)} \\ \hline
			\textbf{M1} Monocristal & 220.4 & 11.4 & 172.4 & 31.3\\ \hline
			\textbf{M2} Monocristal & 227.7 & 11.4 & 180 & 31.2\\ \hline
			\textbf{M3} Monocristal & 232.6  & 11.8 & 167.5 & 28.7\\ \hline
			\textbf{M4} Monocristal & 228.2  & 10.3 & 183.14 & 29.3\\ \hline
			\textbf{M5} Monocristal & 245.5  & 11.5 & 196.3 & 31.7\\ \hline
			\textbf{M6} Monocristal & 239.6  & 11 & 191.5 & 31.9 \\ \hline
			\textbf{M7} Monocristal & 237  & 10.9 & 191.3 & 30.5 \\ \hline
			\textbf{P1} Policristal & 221 & 10 & 176.8 & 33.6 \\ \hline
			\textbf{P2} Policristal & 223.9 & 10.1 & 176.6 & 32.8 \\ \hline	
			\textbf{P3} Policristal & 257 & 12.5 & 209.2 & 32.5 \\ \hline
			\textbf{P4} Policristal & 254.7 & 12.6 & 207.6 & 36.5 \\ \hline
			\textbf{P5} Policristal & 274.6 & 13.2 & 222.6 & 38.5 \\ \hline
			\textbf{P6} Policristal & 268.9 & 12.7 & 217.4 & 37.3 \\ \hline
			\textbf{P7} Policristal & 254.2 & 12 & 205.8 & 33.3 \\ \hline
		\end{tabular}
	\end{center}
\end{table}

\section{Recuperación de Celda solar}
\label{sec:Resultados recuperación de celda}
 
Una vez separados el backsheet y el vidrio, se obtienen dos tipos de muestras: unas mediante un proceso químico y otras a través de un método mecánico. Ambas contienen la celda solar junto con el encapsulante. Para retirar este encapsulante y recuperar la celda solar, se evalúan tanto métodos químicos como mecánicos adicionales. Por ello, se designa como CEQ a las muestras compuestas por celda solar y EVA obtenidas mediante el método químico, y como CEM a las recuperadas mediante el método mecánico, como se mencionó en la sección \ref{sec:Recuperación de celda solar}. Esta nomenclatura tiene como objetivo facilitar la comprensión e identificación de las muestras a lo largo del presente escrito.
 
 
\subsection{Proceso químico para separar EVA de celda solar}
\label{subsec:Resultados metodo quimico celda}

Para separar el EVA, ambas muestras, CEQ y CEM, fueron sometidas a un baño químico con tolueno a una temperatura de 80°C durante 120 minutos. Posteriormente, las muestras fueron centrifugadas a 3500 RPM por 30 minutos. 

\textbf{Muestras recuperadas químicamente (CEQ)}

En la figura \ref{fig:MuestrasPC2yC1}, se observa la muestra policristalina (PC2) y monocristalina (C1) después de centrifugar. Se pudo observar desde el tratamiento químico previo, que el EVA de las muestras policristalinas es menos resistente que el de las monocristalinas. En este caso, se observa que la muestra PC2 tiene leves partículas precipitadas al fondo del recipiente, sin embargo, no se puede asegurar la separación del EVA y celda por diferencia de densidad en el contenedor. En el caso de la muestra C1, se observa el EVA hinchado sin separarse de la celda. 
En ambas muestras, el EVA parece hincharse con el solvente, por lo que se decidió aumentar el volumen del solvente y agregar agua destilada a razón 1:1. 
 

\begin{figure}[htb]
	\begin{center}
		\includegraphics[width=0.2\textwidth]{./Figuras/PC2yC1.jpg}
	\end{center}
	\vspace{-1em} %Para reducir el espacio antes del caption
	\caption{Muestra PC2 y C1 después de tratamiento químico para separar EVA}
	\label{fig:MuestrasPC2yC1}
\end{figure}

\begin{figure}[htb]
	\begin{center}
		\includegraphics[width=0.6\textwidth]{./Figuras/PX3yX3.png}
	\end{center}
	\vspace{-1em} %Para reducir el espacio antes del caption
	\caption{a: muestras PX3 y X3, b: muestras C3 y PC3 después de tratamiento químico}
	\label{fig:MuestrasPX3yX3}
\end{figure}

Las muestras X3 y PX3 fueron tratadas con 14~mL de una mezcla de tolueno y agua destilada en proporción 1:1, mientras que para las muestras C3 y PC3 se utilizó un volumen de 20~mL con la misma proporción. En la Figura~\ref{fig:MuestrasPX3yX3} se muestran las muestras tras el tratamiento. En ambos casos se produce una separación de fases debido a la diferencia de densidades, siendo esta más evidente en las muestras tratadas con 10~mL de la mezcla tolueno-agua (MTA). El EVA flota en la fase de tolueno, mientras que los fragmentos de celda solar se precipitan al fondo del recipiente. Se observa una mejor separación en las muestras policristalinas (PX3 y PC3). 

Dado que en las condiciones iniciales no se logró una separación clara entre el EVA y la celda solar, y en particular en la muestra monocristalina el encapsulante se hinchó pero no se desprendió, se concluyó que esta etapa del método no resultaba efectiva. Por ello, se decidió descartarla como una opción viable para la separación por diferencia de densidad.

\subsection{Proceso mecánico para separar EVA de celda solar}
\label{subsec:Resultados metodo mecanico celda}

\textbf{Muestras recuperadas mecánicamente (CEM)}

Después de retirar mecánicamente el backsheet y el vidrio, se procedió a remover las tiras conductoras de las muestras. Para ello, únicamente fue necesario utilizar unas pinzas y ejercer tracción sobre las tiras. Posteriormente, las muestras fueron trituradas utilizando un molino de acero inoxidable diseñado para granos (Figura \ref{fig:Molino}).   

\begin{figure}[htb]
	\begin{center}
		\includegraphics[width=0.3\textwidth]{./Figuras/Molino.png}
	\end{center}
	\vspace{-1em} %Para reducir el espacio antes del caption
	\caption{Molino para triturar muestras}
	\label{fig:Molino}
\end{figure}

En la figura \ref{fig:Muestratritu} se observan las muestras después de trituración en el molino, la celda se logra triturar y separar, que se aprecia como un polvo fino en la figura. El EVA se tuerce, las cuchillas del molino no lo trituran en partes finas, sino que lo doblan hasta enrollar y  romperse, en esta torsión, la celda se desprende.

Existe una separación del EVA y celda gracias a la trituración (figura \ref{fig:Muestratritu}-a), sin embargo, el EVA aún presenta restos de celda solar.

Si llevamos toda la CEM al proceso químico, la celda triturada será atrapada por el EVA que aumenta su volumen con el solvente. Se puede observar que una vez triturada la CEM, ya existe una separación de componentes debido a que el EVA se deforma pero no se tritura, y en esta torsión, la celda se desprende en partes más finas, por lo que podemos separar estos dos componentes tamizando. Para esto se utilizó un tamiz No. 16 (malla de 1.16mm). 
 
 \begin{figure}[htb]
 	\begin{center}
 		\includegraphics[width=0.7\textwidth]{./Figuras/CEMantesydesp.png}
 	\end{center}
 	\vspace{-1em} %Para reducir el espacio antes del caption
 	\caption{Muestra CEM monocristalina: a. después de triturar, b. CEM después de tamizar}
 	\label{fig:Muestratritu}
 \end{figure}
  
 \begin{figure}[htb]
	\begin{center}
		\includegraphics[width=0.7\textwidth]{./Figuras/CEMEVAseparado.png}
	\end{center}
	\vspace{-1em} %Para reducir el espacio antes del caption
	\caption{EVA separado despues de triturar}
	\label{fig:CEMEVAseparado}
\end{figure}


La figura \ref{fig:Muestratritu}-b, muestra el material de la CEM que pasó por el tamiz, los trozos grandes de EVA que se deformaron por el molino quedan retenidos en el tamiz, estos se muestran en la figura \ref{fig:CEMEVAseparado}, se puede observar que el EVA del panel monocristalino se fragmenta en trozos más pequeños que el policristalino, que solo parece deformarse, el color que presentan es debido al silicio de la celda al momento de triturar. Los pesos de materiales obtenidos en este proceso se muestran en la tabla \ref{tab:PesosEVATamizado}
 
\begin{table}[htb]
	\centering
	\caption{Pesos de EVA separado al triturar y tamizar.}
	\vspace{-0.5em} %Para reducir el espacio después del caption
	\label{tab:PesosEVATamizado}
	\begin{center}
		\begin{tabular}{|>{\centering\arraybackslash}p{3cm}||>{\centering\arraybackslash}p{2.8cm}|>{\centering\arraybackslash}p{2.5cm}|>{\centering\arraybackslash}p{3cm}|>{\centering\arraybackslash}p{3.5cm}|} \hline
			\textbf{Muestra} & \textbf{Peso antes de triturar (g)} & \textbf{contactos (g)} & \textbf{EVA Tamizado (g)} & \textbf{Celda + EVA Tamizado (g)} \\ \hline
			\textbf{M1}  & 34.8 & 2.02 & 17.1 & 12.05\\ \hline
			\textbf{M2}  & 34.4 & 2.49 & 22.4 & 8.4\\ \hline
			\textbf{M3}  & 33.2 & 2.2 & 20.5 & 13.5\\ \hline
			\textbf{M4}  & 29.3 & 3.24 & 18.14 & 9.92\\ \hline
			\textbf{M5}  & 31.7 & 3.37 & 20.86 & 8.74\\ \hline
			\textbf{M6}  & 31.9 & 2.37 & 20.62 & 9.06\\ \hline
			\textbf{M7}  & 30.5 & 2.34 & 16.98 & 8.59\\ \hline
			\textbf{P1}  & 33.6 & 2.29 & 18.4 & 10.71\\ \hline
			\textbf{P2}  & 33.6 & 2.69 & 23.3 & 9.34\\ \hline
			\textbf{P3}  & 32.5 & 2.63 & 21.5 & 8.31\\ \hline
			\textbf{P4}  & 36.5 & 2.66 & 22.3 & 11.46 \\ \hline
			\textbf{P5}  & 38.5 & 2.67 & 23.1 & 12.92\\ \hline	
			\textbf{P6}  & 37.3 & 2.50 & 23.8 & 11.66 \\ \hline
			\textbf{P7}  & 33.3 & 2.54 & 21.2 & 9.22 \\ \hline
		\end{tabular}
	\end{center}
\end{table}
\clearpage

\textbf{Lavado de EVA tamizado}

Las muestras de EVA tamizado se llevaron a agitación con solvente a temperatura ambiente, para evitar el aumento excesivo de su volumen observado en el proceso de baño químico usando 80 °C. El objetivo es lavar el EVA del silicio y trozos de celda que pueda tener. Después es agitado y lavado con agua destilada para quitar restos de solvente. 
 

 \begin{figure}[htb]
	\begin{center}
		\includegraphics[width=0.7\textwidth]{./Figuras/LavadodeEVA.png}
	\end{center}
	\vspace{-1em} %Para reducir el espacio antes del caption
	\caption{Lavado de EVA con xileno y agua}
	\label{fig:LavadoEVA}
\end{figure}


 \begin{figure}[htb]
	\begin{center}
		\includegraphics[width=0.7\textwidth]{./Figuras/EVAquimicoVSlavado.png}
	\end{center}
	\vspace{-1em} %Para reducir el espacio antes del caption
	\caption{Diferencia entre EVA sin tamizar luego de proceso químico con xileno y EVA tamizado después de lavado con xileno y agua}
	\label{fig:EvaqumicoVSlavado}
\end{figure}

La figura \ref{fig:LavadoEVA} muestra el EVA lavado con xileno y posteriormente con agua. El EVA flota en la superficie del recipiente por lo que es recolectado y filtrado, obteniendo un EVA limpio en comparación con el que se obtenía en el baño químico sin tamizar, como se observa en la figura \ref{fig:EvaqumicoVSlavado}.


\textbf{Separación de los restos de EVA de la celda tamizada}

Como se observó en la figura \ref{fig:Muestratritu}, el silicio y metales de la celda encapsulada son separados después de triturar. Algunos restos de EVA logran pasar el tamiz, por lo que para obtener solo la celda triturada se prueban dos métodos: centrifugación y agitación. Ambos se realizan con agua destilada para sedimentar las partículas de celda solar y decantar el EVA. La figura muestra que para ambos el EVA no sube a la superficie para decantar. 
Se propone el uso de tween como agente para propiciar la separación por diferencia de densidades. Se probó tween grado comercial y reactivo en muestras de celda monocristalina triturada con agua razón 6:2. 

                        
\begin{figure}[htb]
	\begin{center}
		\includegraphics[width=0.4\textwidth]{./Figuras/centrifugar_agitar.png}
	\end{center}
	\vspace{-1em} %Para reducir el espacio antes del caption
	\caption{a. Muestra después de centrifugar con agua, b. Muestra después de agitar con agua}
	\label{fig:muestracentrifugadayagitada}
\end{figure}

\begin{figure}[htb]
	\begin{center}
		\includegraphics[width=0.5\textwidth]{./Figuras/tweenreactivoycomercial.png}
	\end{center}
	\vspace{-1em} %Para reducir el espacio antes del caption
	\caption{Muestras después de agitar, a. muestra monocristalina con agua y tween grado reactivo, b. muestra monocristalina con agua y tween comercial}
	\label{fig:tweenreactivovscomercial}
\end{figure}


En la Figura \ref{fig:tweenreactivovscomercial} se muestra que fue posible separar los materiales por diferencia de densidades utilizando una mezcla de agua ($\rho$ = 0.998 g/cm$^{3}$) y tween de grado reactivo ($\rho$ = 1.09 g/cm$^{3}$) en proporción 6:2 en muestras monocristalinas. Al emplear tween comercial en este mismo tipo de muestras, se observa que el silicio precipita mientras una cantidad considerable de EVA permanece flotando en la superficie. No obstante, en el caso del EVA triturado proveniente del panel policristalino, solo unas pocas partículas logran emerger. Es importante señalar que el tween comercial es soluble en agua, lo cual influye en la eficiencia del proceso de separación.

Se decantó el EVA de la muestra monocristalina agitada con agua y tween grado reactivo. De los 4.70 gramos de muestra llevada a agitación, 1.02 g se decantaron de EVA, una vez filtrado el silicio de la solución, se recuperó 3.62 g de este, como se muestra en la figura \ref{fig:celdayevadecantado}.

\begin{figure}[htb]
	\begin{center}
		\includegraphics[width=0.5\textwidth]{./Figuras/celdayevadecantado.png}
	\end{center}
	\vspace{-1em} %Para reducir el espacio antes del caption
	\caption{Muestra monocristalina decantada y filtrada después de agitación}
	\label{fig:celdayevadecantado}
\end{figure}

El tween de grado reactivo demostró ser eficaz para separar el EVA remanente; sin embargo, su alto costo fue uno de los principales inconvenientes y la razón por la cual se consideró el uso de tween de grado comercial. Solo 500 mL del reactivo pueden costar entre 2800 y 4500 MXN, mientras que el tween comercial esta entre 200-500 MXN por litro, sin embargo no mostró eficiencia para separar las partes de la muestra. Si el objetivo es lograr la separación por diferencia de densidades, una alternativa sería aumentar la densidad del agua empleada. \citet{Azeumo2019} propuso una metodología similar, añadiendo sal al agua para incrementar su densidad. No obstante, este enfoque requiere aproximadamente 31 g de sal por cada 10 mL de agua, lo que satura la solución y deja residuos salinos sobre la celda solar, lo cual podría afectar su calidad. 

Como alternativa al uso de sal añadida, se optó por probar agua de mar, aprovechando la cercanía a zonas costeras en Mérida, Yucatán. El uso de agua de mar representa una opción accesible y de bajo costo, lo que favorece la separación por diferencia de densidades entre el EVA y los fragmentos de celda solar. Además, al tratarse de un recurso natural disponible en la región, se elimina la necesidad de agregar y disolver grandes cantidades de sal de forma artificial. Esta solución no solo reduce los costos operativos, sino que también minimiza la cantidad de residuos sólidos remanentes en la celda.

\begin{figure}[htb]
	\begin{center}
		\includegraphics[width=0.5\textwidth]{./Figuras/aguademar.png}
	\end{center}
	\vspace{-1em} %Para reducir el espacio antes del caption
	\caption{Separación de EVA remanente en muestras tamizadas utilizando agua de mar }
	\label{fig:MuestrasAguaMar}
\end{figure}

Los resultados obtenidos al utilizar agua de mar se muestran en la Figura \ref{fig:MuestrasAguaMar}. En ambas muestras, tanto la monocristalina como la policristalina, se observó que el EVA flotó en la superficie, mientras que los fragmentos de celda solar se precipitaron al fondo del vaso. Esta separación clara confirma que el agua de mar tiene una densidad lo suficientemente elevada como para facilitar la separación por diferencia de densidades sin necesidad de aditivos adicionales. 

Las muestras fueron decantadas y filtradas, obteniendo así la celda solar. Los pesos del EVA remanente y celda solar recuperada se muestran en la tabla \ref{tab:PesosEVATamizado}.

En promedio, las muestras monocristalinas presentaron una mayor cantidad de residuo de EVA en comparación con las policristalinas, con un 15.5\% p/p frente a un 5.42\% p/p respecto al peso inicial de cada muestra. Esta diferencia se refleja también en la cantidad de celda solar recuperada: las muestras policristalinas, al tener menos residuo de EVA, permitieron recuperar en promedio un 87.54\% del peso inicial, mientras que en las monocristalinas la recuperación fue ligeramente menor, con un promedio del 80.77\%.

Esta diferencia se observa desde el proceso de triturado. El EVA proveniente de paneles monocristalinos tiende a fragmentarse en trozos más pequeños, mientras que el de los paneles policristalinos suele deformarse sin romperse, manteniendo una forma más compacta. Esto explica por qué las muestras monocristalinas presentaron una mayor cantidad de EVA retenido en el tamiz y, en consecuencia, un mayor porcentaje de residuo.

\begin{table}[htb]
\centering
\caption{Pesos de residuos de EVA y Celda Solar recuperada.}
\vspace{-0.5em} %Para reducir el espacio después del caption
\label{tab:Pesosdecomponentes}
\begin{center}
	\begin{tabular}{|>{\centering\arraybackslash}p{3cm}||>{\centering\arraybackslash}p{2.8cm}|>{\centering\arraybackslash}p{2.5cm}|>{\centering\arraybackslash}p{3cm}|}
		\hline
		\textbf{Muestra} & \textbf{Peso inicial (g)} & \textbf{Residuo de EVA (g)} & \textbf{Celda solar (g)} \\ \hline
		\textbf{M1}  & 5 & 1.13 & 3.67 \\ \hline
		\textbf{M2}  & 4 & 0.78 & 3.07 \\ \hline
		\textbf{M3}  & 4 & 0.14 & 3.77 \\ \hline
		\textbf{M4}  & 5 & 0.91 & 2.96 \\ \hline
		\textbf{M5}  & 5 & 0.44 & 4.15 \\ \hline
		\textbf{M6}  & 4 & 0.70 & 3.19 \\ \hline
		\textbf{M7}  & 4 & 0.61 & 3.23 \\ \hline
		\textbf{P1}  & 5 & 0.35 & 4.62 \\ \hline
		\textbf{P2}  & 4 & 0.14 & 3.47 \\ \hline
		\textbf{P3}  & 4 & 0.10 & 3.83 \\ \hline
		\textbf{P4}  & 5 & 0.34 & 4.62 \\ \hline
		\textbf{P5}  & 5 & 0.50 & 4.10 \\ \hline	
		\textbf{P6}  & 4 & 0.22 & 3.57 \\ \hline
		\textbf{P7}  & 4 & 0.03 & 2.93 \\ \hline
	\end{tabular}
\end{center}
\end{table}


\clearpage
\subsection{Proceso térmico}    
\label{subsec:procesotermico}

El proceso térmico se realizó en un horno tubular marca MTI. La muestra fue colocada dentro de un tubo de cuarzo sellado con tapas de acero inoxidable. Este procedimiento no solo tiene como objetivo la recuperación de silicio, sino también el análisis de los gases emitidos durante la combustión. Para ello, el sistema cuenta con salidas diseñadas específicamente para la recolección de estos gases.

\begin{figure}[htb]
	\begin{center}
		\includegraphics[width=0.9\textwidth]{./Figuras/muestrasprocesotermico.png}
	\end{center}
	\vspace{-1em} %Para reducir el espacio antes del caption
	\caption{Muestras después de tratamiento térmico a)Monocristalina VEVA, b)Policristalina VEVA y c)VEVABS}
	\label{fig:VEVABStermico}
\end{figure}

\begin{table}[htb]
	\centering
	\caption{Parámetros del proceso térmico.}
	\vspace{-0.5em} %Para reducir el espacio después del caption
	\label{tab:rampas proceso térmico}
	\begin{center}
		\begin{tabular}{|c||>{\centering}p{2cm}|>{\centering}p{2cm}|p{2.5cm}|} \hline
		 \textbf{Muestra} & \textbf{Rampa 1} & \textbf{Rampa 2} & \textbf{Temperatura final} \\ \hline
			 \textbf{VEVABS} Vidrio/EVA/Celda/Backsheet	& 24-250°C 30 min & 250-550°C 20 min & 550-850°C 20 min/ 850 1hr\\ \hline
			\textbf{VEVA} Vidrio/EVA/Celda & 24-250°C 30 min & 250-550°C 20 min & 550°C 1hr \\ \hline
			\textbf{EVACS} EVA/Celda & 24-250°C 30 min  & 250-550 20 min & 550°C 1hr \\ \hline
			\textbf{EVA} & 250-550°C 15 min  & - & 550°C 1hr\\ \hline
		\end{tabular}
	\end{center}
\end{table}

La temperatura empleada en el proceso térmico varió según la composición de cada muestra. Para lograr la degradación del EVA fue necesario alcanzar los 500°C, mientras que la eliminación del backsheet requirió temperaturas de hasta 800°C. Se trabajó con muestras de paneles mono y policristalinos, tanto en su forma completa (incluyendo vidrio, celdas, EVA y backsheet) como en versiones pretratadas sin vidrio ni backsheet. Esta diferenciación permitió evaluar el comportamiento térmico de cada componente por separado, así como optimizar el consumo energético del proceso. Las temperaturas específicas y las rampas de calentamiento aplicadas para cada tipo de muestra se detallan en la Tabla\ref{tab:rampas proceso térmico}. 

En la Figura \ref{fig:VEVABStermico} se observan las muestras al finalizar el tratamiento. En particular, para la muestra VEVABS (Figura \ref{fig:VEVABStermico}-c), se observa un cambio de color en la celda solar a tonalidades grises al alcanzar los 850°C, en contraste con las muestras VEVA, donde aún se conserva el característico color azul de las celdas solares.

Una vez concluido el proceso térmico, las bolsas con gases generados fueron enviadas para su análisis mediante cromatografía de gases, y las muestras tratadas fueron extraídas del horno.

\section{Disolución de metales}    
\label{sec:separacióndemetales}		

Una vez separado el EVA, la celda solar es llevada a una digestión ácida con agua regia (mezcla de HCl y HNO$_{3}$ a proporción 3:1) para la disolución de metales presentes como plata, aluminio y residuos metálicos de los contactos eléctricos. 

En la figura \ref{fig:digestion} se observan las muestras con agua regia 40 ml/48 hrs. Debido al costo y a la contaminación del proceso se realizó un número limitado de muestras: 

\begin{itemize}
	\item P2/20ml agua regia-24 hrs. 
	\item P3/20ml agua regia-24 hrs. 
	\item M1/20 ml agua regia-24 hrs.
	\item M2/20 ml agua regia-24 hrs.
	\item P2/40 ml agua regia-48 hrs.
	\item P6/40 ml agua regia-48 hrs.
	\item P5/40 ml agua regia-48 hrs.
	\item M2/40 ml agua regia-48 hrs.
	\item M4/40 ml agua regia-48 hrs
\end{itemize}

Durante el proceso de digestión de las celdas solares, se utilizó una mezcla de agua regia de tonalidad más anaranjada. Este color característico sugiere una alta concentración de especies oxidantes activas y posiblemente la presencia de compuestos intermedios derivados de la interacción entre los ácidos y los materiales metálicos presentes en las muestras. En la figura \ref{fig:digestion} se observan cambios sutiles en la coloración de la solución, lo cual indica una progresiva disolución de los metales, especialmente aquellos visibles en la superficie de las celdas, como plata y aluminio. Durante los primeros minutos de la reacción las muestras policristalinas tuvieron formación de burbujas y un ligero desprendimiento gaseoso (figura \ref{fig:digestionp5}), atribuible a la liberación de gases nitrogenados producto de la descomposición del HNO$_{3}$ en contacto con metales reactivos.

\begin{figure}[htb]
	\begin{center}
		\includegraphics[width=0.9\textwidth]{./Figuras/digestion.png}
	\end{center}
	\vspace{-1em} %Para reducir el espacio antes del caption
	\caption{Muestras en digestión ácida 40 ml-48 hrs}
	\label{fig:digestion}
\end{figure}

\begin{figure}[htb]
	\begin{center}
		\includegraphics[width=0.5\textwidth]{./Figuras/digestionp5.png}
	\end{center}
	\vspace{-1em} %Para reducir el espacio antes del caption
	\caption{Muestra P5 en digestión ácida, primeros segundos en contacto con agua regia}
	\label{fig:digestionp5}
\end{figure}

Después de la digestión ácida las muestras son enjuagadas con agua destilada para eliminar cualquier residuo de agua y filtradas para posteriormente secar a 100°C durante una hora. Estas muestras fueron llevadas a análisis para determinar su composición. 

\clearpage

\section{Análisis de gases emitidos}    
\label{subsec:analisisdegases}

En la sección anterior, se obtuvo la celda solar con tres distintos métodos: químico, mecánico y térmico. El método químico demostró ciertas limitaciones importantes: no se logró una separación eficiente, especialmente en las muestras monocristalinas, donde el EVA permaneció adherido a la celda. Además, la manipulación del solvente a altas temperaturas provocó la formación de geles en muestras policristalinas, dificultando aún más la recuperación del silicio. El método mecánico mostró una clara ventaja al permitir la separación física del EVA mediante trituración y posterior tamizado, obteniendo fracciones de celda y encapsulante. Esta eficiencia se vio incrementada al incorporar etapas adicionales de lavado y separación por densidad. 

Debido a la baja eficiencia del proceso químico y al riesgo de pérdida de material, se descartó este método para las etapas posteriores del estudio. Por esta razón, el análisis de los gases emitidos se llevó a cabo únicamente en dos tipos de muestras: aquellas sin tratamiento previo, descritas en la Tabla \ref{tab:rampas proceso térmico}, y las que fueron sometidas a un tratamiento mecánico antes del proceso térmico. Esta comparación permitió evaluar las diferencias en los gases liberados cuando se aplica únicamente un tratamiento térmico frente a cuando se combinan tratamientos mecánicos como etapa previa. Además, el tratamiento térmico aplicado a las muestras pretratadas mecánicamente cumple también la función de eliminar cualquier residuo de EVA que pudiera permanecer en ellas. 

La Figura \ref{fig:cromatermico} muestra los resultados obtenidos del análisis de los gases generados durante el proceso térmico de muestras sin tratamiento. En todos los casos, se observan picos correspondientes a compuestos detectados en los primeros segundos de tiempo de retención. La Tabla \ref{tab:resultados croma} presenta la identificación de estos compuestos. Aunque en cada muestra solo se detectaron uno o dos compuestos principales, estos tienen un alto impacto tanto en la salud humana como en el medio ambiente.

\begin{table}[htb]
	\caption{Compuestos en gases analizados}
	\vspace{-0.5em} %Para reducir el espacio después del caption
	\label{tab:resultados croma}
	\begin{center}
		\begin{tabular}{|c||c|c|p{9cm}|}\hline
			\textbf{Muestra} &\textbf{Rt} & \textbf{Compuesto} & \textbf{Efectos}\\ \hline
			\multirow{2}{4em}{ \textbf{MVEVABS}}& {1.78} & 2-Buteno & Un gas asfixiante. La evaporación rápida del 2-buteno líquido puede causar congelación. La exposición puede provocar pérdida del conocimiento  \\ 
		   & {2.20} & Benzeno & La exposición humana se ha asociado con una serie de efectos y enfermedades adversos para la salud agudos y a largo plazo, incluidos el cáncer y los efectos hematológicos. \\ \hline
			\textbf{MVEVA} & {1.68} & Pentane 3-Methyl & La exposición humana por inhalación causa síntomas en el sistema nervioso central que incluyen irritación, dolor de cabeza, somnolencia, mareos, pérdida de coordinación, convulsiones y coma.\\ \hline
			\multirow{2}{4em}{ \textbf{PVEVA}} & {0.80} & 2-Buteno & Un gas asfixiante. La evaporación rápida del 2-buteno líquido puede causar congelación. La exposición puede provocar pérdida del conocimiento \\ 
			& {1.68} & 1,4-Hexadieno & Puede irritar los ojos, la garganta y la nariz. Altamente inflamable, puede provocar incendios \\ \hline
			\textbf{EVACS} & {1.69} & Benzeno & La exposición humana se ha asociado con una serie de efectos y enfermedades adversos para la salud agudos y a largo plazo, incluidos el cáncer y los efectos hematológicos. \\  \hline
			\multirow{3}{4em}{\textbf{EVA}} & {0.76} & 2-Buteno & Un gas asfixiante. La evaporación rápida del 2-buteno líquido puede causar congelación. La exposición puede provocar pérdida del conocimiento \\ 
			&{1.24} & \multirow{2}{4em}{Benceno} & La exposición humana se ha asociado con una serie de efectos y enfermedades adversos para la salud agudos y a largo plazo, incluidos el cáncer y los efectos hematológicos.\\  
			&{1.69} &  &  \\  \hline
							\end{tabular}
						\end{center}
					\end{table}

La figura \ref{fig:cromatratadas} muestra los resultados del análisis por cromatografía de gases de muestras de paneles fotovoltaicos monocristalinos y policristalinos que fueron sometidas al tratamiento térmico a 550°C durante una hora. Las gráficas superiores corresponden a muestras que, si bien fueron sometidas previamente a tratamiento mecánico, no pasaron por el proceso de decantación para remover el EVA remanente. En estos casos, se observaron picos bien definidos que indican la presencia de compuestos volátiles: en el monocristalino se detectó 2-hexeno, 5-metil- (E)-, mientras que en el policristalino se identificó benceno, un compuesto tóxico y de interés ambiental.

En contraste, las gráficas inferiores corresponden a las mismas muestras, pero tras haber sido sometidas al proceso de decantación para eliminar el EVA residual. Como puede observarse, en estos cromatogramas ya no se presentan picos definidos, sino únicamente un ruido de fondo, lo cual indica que no se generaron emisiones de compuestos volátiles durante el calentamiento.

Estos resultados demuestran que la eliminación previa del EVA mediante decantación, como parte del tratamiento mecánico, permite reducir eficazmente la presencia de residuos poliméricos responsables de la emisión de gases tóxicos durante el tratamiento térmico. Así, el método mecánico con decantación no solo favorece una separación más limpia de los materiales, sino que también mejora significativamente la seguridad ambiental del proceso al evitar la formación de contaminantes gaseosos. 

\begin{figure}[htb]
	\begin{center}
		\includegraphics[width=0.9\textwidth]{./Figuras/cromatigrafiatermico.png}
	\end{center}
	\vspace{-1em} %Para reducir el espacio antes del caption
	\caption{Resultados de cromatografía de gases}
	\label{fig:cromatermico}
\end{figure}

\begin{figure}[htb]
	\begin{center}
		\includegraphics[width=0.9\textwidth]{./Figuras/GC_contratamiento.png}
	\end{center}
	\vspace{-1em} %Para reducir el espacio antes del caption
	\caption{Resultados de cromatografía de gases en muestras tratadas}
	\label{fig:cromatratadas}
\end{figure}


\clearpage

\section{Evaluación del encapsulante}    
\label{sec:evaluacionEVA}

Con el fin de evaluar la diferencia entre los encapsulantes de los paneles monocristalinos y policristalinos utilizados, fueron recolectadas 2 muestras de EVA previas al tratamiento, una por cada tipo de panel solar. Para esto, se utilizó el procedimiento mecánico para retirar backsheet. Con una pistola de calor se calienta la superficie, a temperaturas entre 50 y 70 °C, parámetros seleccionados de acuerdo con las tablas \ref{tab:PistolaPoli} y \ref{tab:PistolaMono} vistas en la sección \ref{sec:Resultados backsheet}. Después de dpos minutos, el EVA es cuidadosamente retirado la muestra de EVA con ayuda de pinzas. Estas muestras fueron llevadas a FTIR, al igual que EVA comercial sin reticular, para identificar la diferencia entre ellas. 

La figura \ref{fig:FTIR-T} muestra los resultados de FTIR de las muestras pre-tratamiento, se pueden observar picos similares entre las tres muestras. Los EVA presentan transmitancia (T) características del Vinil Acetato (VA) en los números de onda 1740, 1240, 1020 cm$^{-1}$. Para las T en 1470 y 720 cm$^{-1}$ se relacionan con grupos del etileno.

\begin{figure}[htb]
	\begin{center}
		\includegraphics[width=0.7\textwidth]{./Figuras/FTIRcmpAT.png}
	\end{center}
	\vspace{-1em} %Para reducir el espacio antes del caption
	\caption{Espectro FTIR de muestras de EVA antes de tratamientos}
	\label{fig:FTIR-T}
\end{figure}

El contenido de VA en los polimeros es importante porque determina las propiedades de estos. A bajos contenidos de VA, son termoplásticos, y aquellos con mayor contenido son elastomeros. El EVA comercial tiene un porcentaje de VA entre 5 y 40\%. Un EVA tiene buena resistencia a la temperatura y clima entre 12 y 18\% de contenido de VA. El EVA comercial sin reticular no cumple con las propiedades mecánicas ni ópticas de encapsulante, pero cuando se retícula en el proceso de laminación las cadenas del polímero crean un enlace químico transformando el termoplástico inicial en un elastómetro reticulado. En la industria PV esta reacción es activada térmicamente típicamente a 150°C durante la laminación

Se puede relacionar la presencia de VA con los picos característicos en 1740, 1240, 1020 cm$^{-1}$, sin embargo esto no es suficiente para definir el \%VA en las muestras. Una de las propiedades que es inversamente proporcional a la cantidad de VA en el EVA es el \% de cristalinidad (\%C) \citep{Ramirez2019,Henderson1993}. El aumento de \%C esta en función del contenido de la unidad monómerica del etileno, por lo que si el contenido de esta unidad tiende a cero la cristalinidad aumentará a más de 40\%. La señal a 720 cm$^{-1}$ observada en las tres muestras, corresponde principalmente a la flexión o balanceo en el plano de los metilenos (-CH$_{2}$-) del monómero etileno. La señal en 730 cm$^{-1}$, que se observa como un pequeño hombro en 720, indica que parte de las cadenas del polietileno tiene una conformación lineal. 

Observando el espectro FTIR a simple vista, no se encuentra una diferencia en la intensidad de las señales en 720 y 730 cm$^{-1}$. La segunda derivada permite desacoplar dos o más señales de vibración que se superponen permitiendo diferenciar mejor entre espectros. De acuerdo con \citet{Ramirez2019}, la relación entre las áreas debajo de la curva de la señal 730 y 720 $^{-1}$ da una aproximación a \%C, con la ecuación \ref{eq:cristalinidad}. 

\begin{equation}
\label{eq:cristalinidad}
    \%C =\frac{AV}{AC} \times 100
\end{equation}

Donde AV: es el área bajo la curva de la señal en 730 y AC: el área bajo la curva en la señal 720 cm$^{-1}$. Se obtuvo la segunda derivada del espectro FTIR de las muestras de EVA, en la figura \ref{fig:2nderivmono} y \ref{fig:2nderivpoli} se observan las curvas en 730 y 720 cm$^{-1}$, identificando la diferencia entre las curvas en estas señales, los valores de sus áreas bajo la curva se describen en la tabla.

Aplicando la ecuación \ref{eq:cristalinidad}, obtenemos que el EVA monocristalino tiene una cristalinidad aproximada del 2.23\%, mientras que el EVA policristalino del 7.20\%. De acuerdo con \citet{Henderson1993}, entre mayor sea la cristalinidad, menor el contenido de VA, por lo que el EVA de las muestras monocristalinas tiene mayor porcentaje de VA que el EVA de las muestras policristalinas. Por lo que el EVA policristalino se comporta más como un termoplástico que como un elastómetro, eso puede deberse al proceso de reticulación, lo cuál explica porqué es más fácil de separar en el proceso químico. Mientras que el EVA del panel monocristalino con mayor VA, mejora las propiedades del encapsulante y haciendo más difícil la separación del componente en el proceso de reciclaje.

\begin{figure}[htb]
	\begin{center}
		\includegraphics[width=0.8\textwidth]{./Figuras/C_mono.png}
	\end{center}
	\vspace{-1em} %Para reducir el espacio antes del caption
	\caption{a: Segunda derivada del espectro FTIR (730 y 720 cm$^{-1}$ EVA de panel monocristalino}
	\label{fig:2nderivmono}
\end{figure}

\begin{figure}[htb]
	\begin{center}
		\includegraphics[width=0.8\textwidth]{./Figuras/C_poli.png}
	\end{center}
	\vspace{-1em} %Para reducir el espacio antes del caption
	\caption{a: Segunda derivada del espectro FTIR (730 y 720 cm$^{-1}$ EVA de panel policristalino}
	\label{fig:2nderivpoli}
\end{figure}

\begin{table}[htb]
	\centering
	\caption{Áreas debajo la curva en y \%C de muestras monocristalinas y policristalinas antes (S/T) y después del tratamiento químico con cada solvente .}
	\vspace{-0.5em} %Para reducir el espacio después del caption
	\label{tab:cristalinidad}
	\begin{center}
		\begin{tabular}{|>{\centering}p{3cm}||>{\centering}p{3cm}|>{\centering}p{3cm}|c|} \hline
			\textbf{Monocristalino} & \textbf{A 720cm$^{-1}$ } & \textbf{730cm$^{-1}$} & \textbf{\%C} \\ \hline
			ST & 3.7368 & 0.0836 & 2.23 \\ \hline
			Xileno & 4.6626 g & 0.0298 & 0.63 \\ \hline
			Tolueno & 4.5092 & 0.6207 & 1.37 \\ \hline
			Ciclohexano & 4.7485 & 0.0074 & 0.15 \\ \hline
			\textbf{Policristalino} & \textbf{A 720cm$^{-1}$ } & \textbf{730cm$^{-1}$} & \textbf{\%C} \\ \hline
			S/T & 3.7155 & 0.2673 & 7.20 \\ \hline
			Xileno & 3.7107 & 0.0557 & 1.50 \\ \hline
			Tolueno & 3.4282 & 0.8568 & 24.99\\ \hline
			Ciclohexano & 3.5174 g & 0.2099 & 5.96  \\ \hline
		\end{tabular}
	\end{center}
\end{table}   
\clearpage

\subsection{Efecto del solvente en EVA}

Para conocer el efecto en el EVA de cada solvente utilizado en el proceso químico, muestras de panel monocristalino y policristalino se llevaron a baño maría por 120 min a 80 °C con 60 ml de solvente: ciclohexano, xileno y tolueno. Posterior a esto, las muestras se centrifugaron a 3500 RPM por 30 min. Luego se recolecta el EVA de cada muestra y es llevado a FTIR. 

Se observa en la figura \ref{fig:FTIR-sol-mono} que el solvente con mayor efecto en EVA es el tolueno, los picos en longitudes 1018 y 954 cm$^{-1}$ tienen mayor distorsión y menor intensidad, sin embargo el espectro IR se mantiene similar a los de la muestra sin proceso químico vista en la figura \ref{fig:FTIR-T}. En el caso del EVA del panel policristalino (figura \ref{fig:FTIR-sol-poli}), ya no se observan los picos asociados al grupo hidroxilo (-OH) vistos en la figura \ref{fig:FTIR-T}, al contrario, existen picos en el espectro que no se aprecian en la muestra inicial entre las longitudes de  1466 y 607 cm$^{-1}$. Las vibraciones en 607 cm$^{-1}$, corresponden principalmente a los grupos metileno del monómero de EVA, además se observan algunos picos cerca de los característicos al VA en 1740, 1240, 1020 y 610 cm$^{-1}$.  

\textbf{Cristalinidad} 

Para conocer el efecto del solvente en el $\%$C y \%VA, se realizó la segunda derivada del espectro FTIR de las muestras después de tratamiento, como se mostraron en los datos de la tabla \ref{tab:cristalinidad} y se observa en la figura \ref{fig:2ndPoliDP} y \ref{fig:2ndmonoDP}. 
Para el caso de la muestras monocristalinas la cristalinidad disminuye después del tratamiento químico con cada solvente utilizado, siendo el tolueno el que menor reducción presenta. Las muestras policristalinas presentan mayor cambió en su cristalinidad, siendo el caso del tolueno el más notable, aumentando de 1.50\% a 24.99\%. Debido a que el aumento de la cristalinidad indica una disminución en su \%VA y este se relaciona con su calidad; tolueno es el solvente con mayor efecto en la calidad del EVA policristalino. Con esto, se comprueba que el efecto del solvente en el polímero, depende de la calidad inicial de este. A mayor contenido del VA, mayor será la calidad y resistencia al tolueno. 

\begin{figure}[htb]
	\begin{center}
		\includegraphics[width=0.9\textwidth]{./Figuras/FTIRSolventePoli.jpg}
	\end{center}
	\vspace{-1em} %Para reducir el espacio antes del caption
	\caption{a: Espectro IR de muestras policristalinas-EVA después de tratamiento con solvente}
	\label{fig:FTIR-sol-poli}
\end{figure}

\begin{figure}[htb]
	\begin{center}
		\includegraphics[width=0.9\textwidth]{./Figuras/FTIRSolventeMono.jpg}
	\end{center}
	\vspace{-1em} %Para reducir el espacio antes del caption
	\caption{a: Espectro IR de muestras monocristalinas-EVA  después de tratamiento con solvente}
	\label{fig:FTIR-sol-mono}
\end{figure}

\clearpage
\newpage


\begin{figure}[htb]
	\begin{center}
		\includegraphics[width=0.9\textwidth]{./Figuras/C_poliDP.png}
	\end{center}
	\vspace{-1em} %Para reducir el espacio antes del caption
	\caption{Segunda derivada y \%C de muestras policristalinas antes y después de tratamiento con cada solvente}
	\label{fig:2ndPoliDP}
\end{figure}

\begin{figure}[htb]
	\begin{center}
		\includegraphics[width=0.9\textwidth]{./Figuras/C_monoDP.png}
	\end{center}
	\vspace{-1em} %Para reducir el espacio antes del caption
	\caption{Segunda derivada y \%C de muestras monocristalinas. S/N: sin tratamiento y después de tratamiento con cada solvente}
	\label{fig:2ndmonoDP}
\end{figure}

\clearpage
\section{Evaluación de solventes utilizados en proceso químico}    
\label{sec:evaluacionsolventes}

Durante los procesos químicos se utilizaron tolueno y xileno para separar tanto el vidrio como el encapsulante de EVA. Estos solventes son conocidos por su toxicidad y, al reaccionar con el EVA, podrían generar compuestos adicionales que aumenten su peligrosidad. Además, si se pretende reutilizar el solvente, es importante conocer su composición después del proceso. Por esta razón, los solventes utilizados fueron analizados mediante cromatografía de gases para evaluar posibles cambios. En las figuras \ref{fig:ToluenoMonoCromatografia} y \ref{fig:XilenoPoliCromatografia} se muestran los picos de compuestos encontrados en tolueno y xileno usado en el proceso químico de lavado de EVA. La nomenclatura RT es el tiempo de retención del compuesto durante el análisis. Las tablas \ref{tab:cromatografiaxilenopoli} y \ref{tab:cromatografiatoluenopoli}, muestran los principales compuestos encontrados en cromatografía, en ambos casos se muestra como el solvente esta quitando los aditivos plastificantes y agentes de reticulación del polímero. 


\begin{figure}[htb]
	\begin{center}
		\includegraphics[width=0.9\textwidth]{./Figuras/CromatografiaToluenoMono.png}
	\end{center}
	\vspace{-1em} %Para reducir el espacio antes del caption
	\caption{Resultados de cromatografía de gases, tolueno usado en proceso lavado de EVA panel monocristalino}
	\label{fig:ToluenoMonoCromatografia}
\end{figure}

\clearpage

\begin{table}[htb]
	\caption{Compuestos encontrados en cromatografía de gases: Tolueno en muestra monocristalina.}
	\vspace{-0.5em} %Para reducir el espacio después del caption
	\label{tab:cromatografiatoluenopoli}
	\begin{center}
		\begin{tabular}{|p{3cm}||p{4cm}|p{5cm}|}\hline
			\textbf{RT} & \textbf{Compuesto} & \textbf{Uso}\\ \hline
			\textbf{42.83} & 1,4-Benzenedicarboxylic acid, bis(2-ethylhexyl) ester & Plastificante compatible con resinas de acetato \\ \hline
			\textbf{39.95} & Phthalic acid, 6-ethyloct-3-yl 2-ethylhexyl ester & Aditivo para mejorar la extensibilidad y flexibilidad \\ \hline
			\textbf{30.80} & Octadecane, 3-ethyl-5-(2-ethylbutyl)- & Antimicrobianos eficaces y agentes antifúngicos. \\ \hline
			\textbf{24.10} & 1,3,5-Triazine-2,4,6(1H,3H,5H)-trione, 1,3,5-tri-2-propenyl- & Agente acelerador de reticulación para peróxido\\ \hline	
		\end{tabular}
	\end{center}
\end{table}


\begin{figure}[htb]
	\begin{center}
		\includegraphics[width=0.9\textwidth]{./Figuras/XilenoCromatografiaPoli.png}
	\end{center}
	\vspace{-1em} %Para reducir el espacio antes del caption
	\caption{Resultados de cromatografía de gases, Xileno usado en proceso lavado de EVA panel policristalino}
	\label{fig:XilenoPoliCromatografia}
\end{figure}


\begin{table}[htb]
	\caption{Compuestos encontrados en cromatografía de gases: Xileno en muestra policristalina.}
	\vspace{-0.5em} %Para reducir el espacio después del caption
	\label{tab:cromatografiaxilenopoli}
	\begin{center}
		\begin{tabular}{|p{3cm}||p{4cm}|p{5cm}|}\hline
			\textbf{RT} & \textbf{Compuesto} & \textbf{Uso}\\ \hline
			\textbf{9.48} & Benzaldehyde, 4-methyl- & Fijador de aromas, también se utiliza en la síntesis de productos farmacéuticos  \\ \hline
			\textbf{40} & Phthalic acid, di(2-propylpentyl) ester & Plastificante y aditivo para mejorar la extensibilidad y flexibilidad\\ \hline
			\textbf{42.85} & 1,4-Benzenedicarboxylic acid, bis(2-ethylhexyl) ester & Plastificante \\ \hline
			\textbf{24.09} & 1,3,5-Triazine-2,4,6(1H,3H,5H)-trione, 1,3,5-tri-2-propenyl- & Agente acelerador de reticulación para peróxido\\ \hline	
		\end{tabular}
	\end{center}
\end{table} 

Con los resultados obtenidos se observa que tanto el Xileno como Tolueno produce la extracción de diversos compuestos del polímero. La presencia de estos indica que no solo disuelven la matriz polimérica, sino también los aditivos, confirmando su efectividad. Sin embargo pueden modificar las características y toxicidad del solvente residual representando desafío ambiental, ya que los solventes contaminados con plastificantes y productos de degradación del EVA requieren un manejo adecuado para evitar impactos negativos en su posible reutilización o disposición final.

\clearpage
 \section{Análisis de energía utilizada en el proceso de reciclaje}    
\label{sec:energiadelproceso}

Las figuras \ref{fig:energiaruta1} y \ref{fig:energiaruta2}, muestran los procesos, equipos, tiempo y energía utilizados en cada etapa de las dos rutas de reciclaje aplicadas en este trabajo. En la ruta 1, se toman en cuenta dos escenarios: sin separar EVA con tratamiento químico y con tratamiento químico para retirar EVA. Así mismo, la ruta 2 considera escenarios sin separar EVA y solo utilizar tratamiento térmico, otro separando el EVA y obteniendo la celda por método mecánico, y un último en el que se combinan ambos. En ambas rutas, el separar EVA aumenta el tiempo y la energía utilizada. Se observa que la Ruta 2 es significativamente más eficiente. Para el caso del silicio monocristalino, la energía total consumida en la Ruta 1 es de 6599 Wh, mientras que en la Ruta 2 se reduce a 4605 Wh. Una diferencia similar se presenta en el caso del policristalino, donde se pasa de 6605 Wh en la Ruta 1 a 4761 Wh en la Ruta 2. 

La ruta 2, se demostró como la más eficiente para la obtención de componentes así como respecto al consumo energético. Utilizar solo el método térmico para la recuperación de silicio, es el proceso más rápido, sin embargo como se describió en la sección \ref{subsec:analisisdegases}, su principal desventaja es la emisión de gases tóxicos como bencenos y alquenos, vistos en los cromatogramas. El separar el EVA mecánicamente, presenta un enfoque más balanceado, aunque el tiempo y energía utilizada aumentan, se logra recuperar la Celda y EVA, sin la emisión de gases tóxicos, así mismo, el combinar este proceso con el térmico asegura la eliminación total de residuos en la celda manteniendo un consumo energético moderado. 

\begin{figure}[htb]
	\begin{center}
		\includegraphics[width=1\textwidth]{./Figuras/energiaRuta1.png}
	\end{center}
	\vspace{-1em} %Para reducir el espacio antes del caption
	\caption{Tiempo y energía utilizada en cada etapa del reciclaje ruta 1}
	\label{fig:energiaruta1}
\end{figure}

\begin{figure}[htb]
	\begin{center}
		\includegraphics[width=1\textwidth]{./Figuras/EnergiaRuta2.png}
	\end{center}
	\vspace{-1em} %Para reducir el espacio antes del caption
	\caption{Tiempo y energía utilizada en cada etapa del reciclaje ruta 2}
	\label{fig:energiaruta2}
\end{figure}

\section{Análisis de costo del proceso de reciclaje}    
\label{subsec:costodelproceso}

En la sección anterior, se obtuvo la energía utilizada en cada ruta propuesta. En la presente sección, se calcula el costo de cada proceso utilizado. Los resultados se muestran las figuras \ref{fig:costosruta1} y \ref{fig:costosruta2}. En el caso de la Ruta 1, separar el EVA aumenta más del 50\% el costo del proceso. Utilizando solo el método térmico, el precio total del proceso es de  \$ 41.61 MXN, el aumento de este proviene del horno para la degradación más el costo de los solventes del método químico, con un total de \$ 104.50 MXN. 

La Ruta 2 se muestra más eficiente tanto en términos económicos como operativos. En esta ruta, la separación del EVA se realiza mediante un método mecánico basado trituración, tamizado y decantación de residuos a base de diferencia de densidades con agua de mar, cuyo costo es apenas \$4.37 MXN. El tratamiento térmico complementario para eliminar residuos tiene un costo de \$10.28 MXN, pero al no requerir solventes, el costo total del proceso completo es de apenas \$10.50 MXN, lo que representa una reducción de casi 90\% en comparación con la Ruta 1. Esta ruta no solo reduce los costos drásticamente, sino que también mantiene la eficiencia del proceso al lograr una adecuada eliminación del EVA sin generar emisiones peligrosas, como se evidenció en el análisis de cromatografía de gases previo. Por tanto,  representa una opción más viable y ambientalmente segura para el reciclaje de módulos fotovoltaicos.


\begin{figure}[htb]
	\begin{center}
		\includegraphics[width=1\textwidth]{./Figuras/costosruta1.png}
	\end{center}
	\vspace{-1em} %Para reducir el espacio antes del caption
	\caption{Precio en MXN del proceso de reciclaje Ruta 1, por muestras de 10x10 cm}
	\label{fig:costosruta1}
\end{figure}



\begin{figure}[htb]
	\begin{center}
		\includegraphics[width=1\textwidth]{./Figuras/costosruta2.png}
	\end{center}
	\vspace{-1em} %Para reducir el espacio antes del caption
	\caption{Precio en MXN del proceso de reciclaje Ruta 2, por muestras de 10x10 cm}
	\label{fig:costosruta2}
\end{figure}


