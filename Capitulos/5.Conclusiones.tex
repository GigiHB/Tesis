 En este capitulo se presentan las principales conclusiones obtenidas a partir del análisis de resultados vistos en el capitulo \ref{ch:Resultados}, destacando los hallazgos relevantes en cada etapa del proceso de reciclaje de panales solares y su impacto ambiental.
 
 \section{Separación de componentes}
 \label{sec:Separacion de compoenentes conclusiones}
 
 La evaluación de métodos para la separación de los componentes en módulos fotovoltaicos mostró diferencias significativas entre paneles monocristalinos y policristalinos. Principalmente por la composición y calidad del encapsulante usado para cada tipo de panel. Cabe resaltar que, si bien en este trabajo se identificó el tipo de muestra utilizada como monocristalina o policristalina, estas provienen de diferentes fabricantes. Por lo tanto, la diferencia en la calidad de sus componentes depende únicamente de la selección de materiales de cada marca, y no de la cristalinidad de la celda solar. 
 
 \subsection{Separación de backsheet}
 \label{subsec:Conclusiones Backsheet}

 Para los paneles monocristalinos, la temperatura ideal para separar el backsheet sin daños ni residuos está entre 100 y 130°C con un flujo de aire moderado (300 L/min), mientras que para policristalinos se requiere menor temperatura (60-70°C) y mayor velocidad de aire (500 L/min) para evitar el deterioro térmico del EVA.
 El método mecánico empleado logró la completa separación del backsheet para ambos tipos de muestra. 
 
 \subsection{Obtención de vidrio}
 \label{subsec:Conclusiones vidrio}
 
 Se probaron dos métodos para la recuperación de vidrio: químico y mecánico. Los resultados del método químico mostraron que la selección del solvente y calidad del EVA en el panel son factores críticos para la optimización del proceso. Los resultados de este método fueron variables dependiendo del solvente y muestra empleada. Para paneles policristalinos xileno y tolueno permitieron una separación más eficiente. En cambio, en paneles monocristalinos el xileno solo logró una separación parcial, con EVA aún adherido al vidrio. La presencia de EVA y celda solar en las muestras de vidrio después del proceso, sugieren que este método requiere ajustes en los solventes o condiciones experimentales para mejorar la pureza del vidrio recuperado. 
 
 La separación mecánica mediante pistola de aire caliente y uso de cuchilla demostró ser un método efectivo para separar físicamente el vidrio del EVA y la celda. Esta técnica puede ser aplicada de manera rápida y con menos uso de químicos, ofreciendo una alternativa viable para la recuperación de vidrio en ambos tipos de paneles.
 
 \subsection{Recuperación de Celda Solar}
 \label{subsec:Conclusiones recuperación de celda}
 
 Las muestras resultantes del proceso anterior dividieron en: obtenidas químicamente (CEQ) y las recuperadas mecánicamente (CEM). La separación del encapsulante EVA de las celdas solares presenta diferencias notables según el tipo de muestra. 
 
 El método químico a 80 °C mostró que en ambas muestras el EVA se hincha en contacto con el solvente sin lograr una separación efectiva. Aunque la adición de agua a razón 1:1 mejoró la separación por diferencia de densidades, esta fue mínima. Las muestras CEQ fueron las que mostraron menor separación, tras estar previamente en contacto con solventes, el EVA de estas muestras se conglomeró junto con los trozos de la celda solar haciendo imposible su separación. Aunque las muestras CEM no mostraron estos conglomerados de EVA y celda, se presentó el efecto de hinchado de EVA que no permitió la separación de los componentes. Debido a esto el proceso fue descartado. 
 
 La trituración mecánica mediante molino permitió que la celda se desprendiera del encapsulante obteniendo EVA, que se deforma y enrolla durante la trituración, mientras que celda se fragmenta en partes finas. Esto facilitó la separación de los componentes con tamizado, recuperando ambas fracciones.  
 De la fracción que pasó el tamiz, compuesta principalmente por finas partes de celda solar, aún permanecieron residuos de EVA que lograron pasar el tamiz. La separación del EVA remanente se logró mediante diferencia de densidades y decantación. Se evaluó la separación con tween grado reactivo y comercial, pero debido al alto costo del tween reactivo y la ineficiencia del comercial, se utilizó agua de mar, recurso disponible y de fácil acceso por cercanía del proyecto a aguas costeras. Con la agitación de la muestra por unos minutos, se logró la precipitación de la celda mientras el EVA, por diferencia de densidad, flotó en la superficie del agua de mar. 
 El material que no logró pasar el tamiz, fue el EVA que se deformó y enrolló durante la trituración. El cambio de color en el polímero a gris se atribuye al contacto con la celda al momento de romperse en partes finas. El EVA lavado con xileno removió los residuos de silicio y metales, hacer el lavado químico a temperatura ambiente evitó que el polímero se hinchara como se observó al utilizar temperaturas de 80 °C, obteniendo un EVA más limpió y con la posibilidad de su reciclaje. 
 
 El tratamiento térmico es uno de los más utilizados. Asegura la obtención de silicio y metales, si bien ese es el objetivo principal, en este trabajó más que evaluar la recuperación de celda se realizó con el fin de recolectar los gases emitidos durante el proceso para posteriormente ser analizados por cromatografía de gases. Se trabajaron con muestras representativas de cada panel sin tratamiento. Debido a que la ruta 1 fue descartada por al ineficiencia del método químico para la obtención de la celda, el tratamiento térmico solo se realizó para muestras obtenidas mecánicamente con el fin de evaluar la diferencia entre los gases emitidos. 
 El proceso térmico permitió la degradación efectiva del encapsulante (EVA) a temperaturas superiores a 500 °C, mientras que la eliminación completa del backsheet requirió temperaturas de hasta 850 °C. Esto confirma que los distintos componentes del módulo fotovoltaico tienen comportamientos térmicos diferenciados que deben considerarse al diseñar un proceso de reciclaje eficiente.  
 Las muestras VEVABS tratadas a 850°C presentaron un cambio notable en la coloración de las celdas solares, indicando una posible modificación en la estructura del silicio debido a la alta temperatura, lo que puede impactar la calidad del material recuperado. Las muestras VEVA mantenidas a 550°C conservaron el color azul característico, sugiriendo un menor impacto sobre la celda.
 El sistema de horno tubular con recolección de gases permitió la captura de los gases generados durante el proceso para su posterior análisis de composición. 
 
 
 En conjunto, la ruta 2 para la obtención de la celda (método mecánico basado en trituracion, tamizado y decantación) se presenta como un proceso viable y eficiente. Este, permite recupera tanto celda solar como el encapsulante (EVA), material que en procesos convencionales es comúnmente desechado o degradado. En comparación con el método térmico, esta ruta reduce el gasto energético y elimina la necesidad de de sistemas de captación o tratamiento de gases, simplificando el proceso y mitigando la emisión de gases y residuos tóxicos de los métodos térmicos. 
 
 \subsection{Disolución de metales}
 \label{subsec:Conclusiones disolución de metales}
 
 La digestión ácida con agua regia se realizó para obtener el silicio de la celda, debido al costo y contaminación inherente de agua regia se realizó un número limitado de muestras representativas para validar la recuperación de silicio y metales. 
 El color anaranjado inicial del agua regia indicó alta concentración de especies oxidantes activas que favorecen la disolución de los metales. El cambio de color progresivo de la solución con las muestras evidenció que se disolvieron eficazmente los metales de la celda solar. 
 Las muestras policristalinas mostraron mayor actividad al contacto con la solución asociada con la liberación de gases nitrogenados por la descomposición del ácido nítrico en contacto con metales. 
 El proceso de lavado con agua destilada y secado permite eliminar residuos para su preparación adecuada para el análisis de composición.
  
\section{Evaluación de gases emitidos durante el proceso térmico}
\label{sec:Conclusiones gases emitidos}
 
 Se realizó el análisis de cromatografía de los gases recuperados tras un tratamiento térmico tanto para muestras después del proceso mecánico, como para muestras representativas de cada etapa (MVEVABS: muestra de panel inicial, MVEVA: monocristalina son backsheet, PVEVA: policristalino sin backsheet, EVACS: muestras sin vidrio y backsheet y EVA). Todas la muestras sin tratamiento mecánico previo emitieron gases que representan un riesgo tanto para el medio ambiente como para la salud humana. Se identificaron gases como 2-Buteno y Benzeno, asociados con una serie de efectos y enfermedades adversos a la salud y toxicidad LD50 de 1190 mg/kg y 930 mg/kg cada uno. Al igual que gases irritantes como Pentane 3-Methyl y 1,4-Hexadieno con toxicidad LD50 de 2000 mg/kg y 150 mg/kg para cada uno.  
 En el análisis realizado a los gases emitidos de muestras pre-tratadas mecánicamente no se presentaron picos definidos en los cromotagramas, demostrando que el método mecánico reduce significativamente la emisión de gases volátiles. 
 Estos resultados indican que la combinación de métodos mecánicos y térmicos mejora no solo la eficiencia de separación, sino también la seguridad ambiental del proceso, minimizando la liberación de contaminantes gaseosos durante el reciclaje de celdas solares.
 
 \section{Evaluación del Encapsulante}
 \label{sec:Conclusiones encapsulante}
 
 La cuantificación relativa de cristalinidad (\%C), determinada a partir de la segunda derivada del espectro FTIR mostró que el EVA de los paneles policristalinos posee mayor cristalinidad ($~$7.20\%) que el EVA de los monocristalinos ($~$2.23\%). Esto indica un menor contenido de VA en el EVA policristalino. Esto indica que el EVA del panel policristalino tiene un comportamiento característico de termoplástico, lo que facilita su separación en procesos químicos de reciclaje. El EVA monocristalino, con mayor contenido de VA, se comporta más como elastómero, otorgándole mejores propiedades mecánicas y ópticas, pero dificultando su separación. Estas diferencias influyen directamente en la eficiencia y dificultad de los procesos de reciclaje.
 
 Se realizó el mismo análisis para muestras de EVA después de la exposición con tolueno, xileno y ciclohexano. De acuerdo con el resultado de FTIR, el tolueno presentó mayor efecto en la estructura, especialmente en muestras policristalinas donde se observó un aumento de la cristalinidad (de 1.50 \% a 24.99\%) lo que indica un reducción en el contenido de VA, afectando la calidad del polímero. En el EVA monocristalino, el cambio en cristalinidad después del tratamiento fue menos marcado, provocando la menor reducción, lo que refleja una mayor resistencia química atribuible al mayor contenido de VA.
 Estos resultados marcan que el efecto del solvente en la estructura del EVA depende fuertemente de la calidad inicial del polímero; a mayor contenido de VA, mayor resistencia a la degradación por solventes.
 
 \section{Evaluación de solventes}
 \label{sec:Conclusiones solventes}
 
 Los análisis por cromatografía de gases demostraron que los solventes empleados son efectivos para la extracción de los aditivos plastificantes y agentes de reticulación presentes en el EVA. La modificación química de los solventes debido a los compuestos extraídos representa un reto ambiental importante, pues aumenta la toxicidad y complejidad del manejo del solvente después de su uso, afectando su posible reutilización o disposición final segura. 
 Se recomienda implementar estrategias para el tratamiento y manejo adecuado de los solventes usados, para mitigar riesgos ambientales y asegurar la sostenibilidad del proceso químico empleado en el reciclaje de paneles fotovoltaicos. 
 
 \section{Análisis de energía empleada en el proceso de reciclaje}
 \label{sec:Conclusiones energia empleada}  
 
 La separación del encapsulante EVA, ya sea por métodos químicos o mecánicos, incrementa el tiempo y la energía requerida, sin embargo, la combinación de métodos mecánico y térmico ofrece un balance adecuado entre eficiencia energética y reducción de emisiones contaminantes. El uso exclusivo del método térmico es el proceso más rápido y de menor consumo energético, pero presenta una desventaja ambiental relevante por la emisión de gases tóxicos (bencenos y alquenos), evidenciada en el análisis cromatográfico.
 
 El análisis energético de las dos rutas de reciclaje evaluadas demuestra que la Ruta 2 es significativamente más eficiente en consumo de energía que la Ruta 1. Integrar la separación mecánica del EVA con tratamiento térmico posterior resulta una alternativa más sostenible y equilibrada, que permite recuperar los componentes con un consumo energético moderado y menor impacto ambiental.
 
 \section{Análisis de costos del proceso de reciclaje}
 \label{sec:Conclusiones costos del proceso}
 
 En la Ruta 1, la separación del EVA mediante tratamiento químico y térmico incrementa el costo total a \$104.50 MXN, principalmente debido al gasto energético del horno y al uso de solventes, que además representan un riesgo ambiental.
 La Ruta 2 emplea un método mecánico sencillo y de bajo costo para separar el EVA, combinado con un tratamiento térmico complementario, logrando un costo total aproximado de \$10.50 MXN
 
 La separación mecánica basada en trituración y decantación con agua de mar es un método económico, y al evitar el uso de solventes químicos se reducen los gastos asociados al manejo de residuos peligrosos.
 