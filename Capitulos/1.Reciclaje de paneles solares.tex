
\section{Introducción}
\label{sec:Introducción}

Debido al incremento de la demanda de energía, emisiones de CO$_{2}$, regulaciones ambientales, y el aumento del precio del petróleo, la producción de energía por combustibles fósiles debe ser sustituida por la generación a partir de fuentes renovables. La energía solar fotovoltaica es un tipo de energía renovable que ha ganado reconocimiento por ser amigable con el medio ambiente, no produce ruido ni gases tóxicos, y elimina el alto costo ambiental que genera el quemar combustibles fósiles \citep{Kim2012}. 

La energía solar fotovoltaica inició en 1954, cuando se desarrolló la primera celda solar de silicio monocristalino como material fotovoltaico, esta tecnología de celdas solares es denominada como de primera generación. La segunda generación de celdas solares se basa en películas delgadas de materiales como Teluro de Cadmio (CdTe) y las denominadas CIGS (Cobre, Indio, Galio, Selenio), mientras que las de tercena generación son celdas que permiten eficiencias teóricas mucho mayores que las actuales y a un precio menor, es una tecnología emergente que utiliza materiales como polímeros, así como híbridos orgánico-inorgánico y perovskitas \citep{Oyola2007}.

Al conjunto de celdas solares agrupadas se le denomina módulo fotovoltaico, estas celdas transforman la energía solar en energía eléctrica,  Una celda solar básica es una unión PN con un contacto en la región P y otro en la región N que permiten la conexión con un circuito eléctrico. Las convencionales están construidas a partir de una oblea de material semiconductor como el silicio, de un espesor aproximado de entre 100 y 500 micrometros, en la que se ha difundido una impureza trivalente, región P y sobre la que se difunde una capa muy fina, de 0,2 a 0,5 micrometros, de una impureza pentavalente, región N, para obtener una unión PN. La mayoría de los módulos fotovoltaicos tienen entre 36 y 96 celdas conectadas en serie. Además, tienen una protección frente a los agentes atmosféricos, un aislamiento eléctrico adecuado y una consistencia mecánica que permita su manipulación práctica \citep{Florencia2013}.

La producción e instalación de paneles fotovoltaicos ha crecido significativamente en todo el mundo, entre el 2015 y 2019 el total de energía fotovoltaica instalada creció un 62\% \citep{IRENA2019}, esto es debido al incremento de la eficiencia de las celdas y la disminución en costos de producción. Actualmente, las celdas de silicio cristalino dominan el mercado fotovoltaico, con una participación de 85\% de las tecnologías que lo conforman. Se espera que este material continúe como líder en el desarrollo de tecnologías fotovoltaicas por lo menos durante la siguiente década \citep{Oyola2007}. Las celdas de silicio monocristalinas muestran eficiencias hasta de 26.1\%, mientras que las celdas de silicio policristalino alcanza hasta un 23.3\% \citep{NREL2020}

Las instalaciones de paneles fotovoltaicos para uso domésticas van desde 3kW conectados a la red y hasta 450MW para plantas abastecedoras a las redes eléctricas de transmisión, registrando una capacidad mundial instalada de 222 GW  a finales del año 2015 \citep{Pro2017}  y para el 2019 la capacidad instalada de energía solar fotovoltaica global ya era de 580.59 GW \citep{IRENA2019}. La cantidad de paneles solares crece rápidamente y se espera que incremente aún más en los siguientes años. Normalmente la vida útil de un panel solar es de 25 años, se calcula que para el año 2050 se tengan 9.57 millones de toneladas de paneles solares desechados \citep{Xu2012}. Por lo tanto, es importante elaborar un plan para reciclar futuros residuos fotovoltaicos, ya que el reciclaje tiene grandes beneficios como la recuperación y reutilización de materiales. Los módulos fotovoltaicos utilizan metales valiosos como el oro, plata, teluro, indio, galio y otros materiales como vidrio y aluminio capaces de ser recuperados, reciclados y reutilizados dentro o fuera de la misma industria fotovoltaica \citep{Dominguez2017}.

En México las primeras instalaciones fotovoltaicas fueron utilizadas principalmente para la electrificación de comunidades rurales, para el año 2013 se tuvo un gran incremento en la capacidad instalada (82 MW) debido a la construcción de la primera planta solar Aura 1 \citep{Dominguez2017}. De acuerdo con el SIE en el 2018 México contó con una capacidad de 23.98 petajoules en energía solar fotovoltaica \citep{SIE2019}. Este crecimiento de fotovoltaica en México apenas comienza, y aún no se cuentan con regularizaciones para el manejo de los residuos al finalizar su vida útil \citep{Dominguez2017}. En la Unión Europea, al final de su vida útil, los paneles solares son clasificados como residuos de aparatos eléctricos y electrónicos (WEEE por sus siglas en inglés), mientras que en otros países son parte de la clasificación general de residuos \citep{Azeumo2019}.

Recientemente se han realizado avances en investigaciones de reciclaje de paneles solares, principalmente en Europa, Japón y Estados Unidos. La mayoría de los estudios están enfocados en la extracción del silicio y reciclaje de metales raros \citep{Xu2012}. Los módulos fotovoltaicos están conformados por una capa de vidrio, marco de aluminio, una capa que encapsula la celda de EVA (Ethyl Vinil Acetate), la celda fotovoltaica, una caja de conexiones y un polímero denominado Backsheet que da soporte a las celdas. El marco de aluminio es separado del panel mecánicamente, el vidrio puede ser recuperado por procesos térmicos o mecánicos mientras que el silicio y metales son separados por métodos tanto térmicos como químicos.  Para cualquier proceso de recuperación de silicio y metales es necesario eliminar el EVA y Backsheet de la celda fotovoltaica, existen métodos tanto térmicos, químicos y mecánicos para esto. Se ha visto que los más eficientes son los químicos y térmicos, pero aún no se tiene con exactitud una estimación de la emisión de gases tóxicos e impacto ambiental que dejan estos \citep{Fiandra2019}. En el presente trabajo, se realizará un estudio del método químico para eliminar el EVA, haciendo un análisis de los residuos químicos que resultan de este primer paso en el proceso de reciclaje de módulos fotovoltaicos. Es importante buscar métodos eficientes que minimicen los impactos ambientales. 

\section{Antecedentes}
\label{sec:Antecedentes}
Actualmente, se está llevando a cabo investigaciones sobre el reciclaje de paneles solares, principalmente en Europa, Japón y Estados Unidos. La mayoría de los estudios de reciclaje de paneles solares se han centrado en la extracción de silicio y el reciclaje de elementos metálicos. En la actualidad, hay tres métodos de procesamiento de residuos de paneles solares: reparación de componentes, recuperación de la celda fotovoltaica y recuperación de silicio y otros elementos de metales raros \citep{Xu2012}. Todos los métodos de reciclaje tienen un proceso en común, la eliminación del EVA y Backsheet. Debido a que no son reciclables se emplean procesos térmicos y químicos para su eliminación. 

\citet{Doi2001}, prescribió la recuperación de celdas fotovoltaicas de paneles solares de silicio cristalino convencionales empleando un solvente orgánico. En las pruebas  utilizan múltiples tipos de solventes orgánicos para disolver la película de etilvinilacetato (EVA), se descubrió que el tricloroetileno podía fundir muestras de EVA reticuladas mantenidas a 80 $^{\circ}$C. Aplicando este método a un panel solar se encontró que la presión mecánica es necesaria para suprimir la hinchazón del EVA. Después de sumergir el módulo en tricloroetileno a 80 $^{\circ}$C durante diez días, se eliminó el EVA y la celda de silicio se recuperó con éxito sin ser dañada \citep{Doi2001}.

Se han reportado tratamientos térmicos para eliminar EVA y Backsheet. Para la degradación del EVA se necesitan temperaturas de 500 a 600$^{\circ}$C durante una hora para su completa eliminación \citep{Pagnanelli2019}. \citet{Farneth2012}, estudiaron la degradación de varios polímeros, entre ellos EVA y Tedlar™ (polímero del Backsheet), concluyendo que para la completa degradación del Tedlar™ se necesitan 800$^{\circ}$C, y que en el proceso de descomposición de la cadena polímera, emite grandes cantidades de gases tóxicos \citep{Farneth2012}. Debido a esto otros autores llevan la celda a un pre-tratamiento térmico de hasta 330$^{\circ}$C por tiempos no mayores a 30 min, para posteriormente separar físicamente el backsheet y así evitar la emisión de grandes cantidades de gases tóxicos \citep{Wang2012}.

Para la degradación del EVA también han estudiado los efectos de la concentración de diferentes solventes a distintas temperaturas, potencia ultrasónica, tiempos de radiación, etc. El efecto en la degradación del EVA depende del tipo de solvente utilizado, su concentración y tiempo sumergido. Los solventes orgánicos más eficientes para la degradación del EVA son el tolueno, tricloroetileno y diclorobenceno, como lo demostró \citet{Sukmin2007} donde experimentó la disolución del EVA con distintos solventes, encontrando estos últimos mencionados como los mejores para retirar el EVA de la celda. Sin embargo, el proceso utilizado llevó un largo periodo (2días) y temperaturas de 90$^{\circ}$C para lograr que el EVA se diluyera en el solvente. Posteriormente se procedió a un tratamiento térmico, debido a que aún se detectaron restos de EVA despúes del tratamiento químico. \citep{Sukmin2007}. 

\citet{Kim2012} mostraron que a una potencia ultrasónica de 450 W y una temperatura de 70$^{\circ}$ C, la película de EVA puede disolverse por completo en 3 mol/L de tolueno después de una hora. Este método acorta significativamente el tiempo de disolución de EVA en disolventes orgánicos. Sin embargo, al llevar la celda completa a altas potencias ultrasónicas provocó que se rompiera parcialmente, por lo que proponen mejorar la potencia para poder recuperar la celda sin que esta sufra daños. Algunos solventes tienen altos costos, y usar equipos ultrasónicos aumenta aún más el precio del proceso, esto es uno de los motivos que hace que los tratamientos químicos sean menos usuales que los térmicos, \citep{Kim2012}.

Los métodos químicos son comunmente usados para recuperar la celda completa y así poder ser reutilizada, como se ha reportado, uno de los problemas de esto es que en el proceso la celda tiende a romperse o perder calidad por las altas concentraciones de solventes utilizados. \citet{Azeumo2019}, estudiaron un método físico-químico para la obtención del silicio y metales. El procesos físico, consiste en la separación de componentes, primero las muestras fueron cortadas en tamaños de 7x1.1x1cm para posteriormente ser fragmentadas con un molino de cuchillas, luego estas muestras fueron pasadas por rejillas de 0.4cm, lo que pasaba la rejilla fue recolectado y los restos llevados de nuevo al molino hasta obtener el tamaño deseado. Estas muestras molidas fueron llevadas al procesos de separación por peso, en el que primero se les añadió agua, posteriormente cloruro de sodio con densidad de 1.2g/cm$^{3}$, y por último politungstato de sodio con densidad de 1.5g/cm$^{3}$. Con la diferencia de densidad de los componentes los metales fueron recolectados manualmente. El proceso químico consistió en la separación de vidrio y EVA, para esto las muestras fueron puestas en un matraz con solvente orgánico. El matraz es sumergido en baño utrasónico a temperaturas de ebullición del solvente. Se contó con un sistema de recirculación de agua con el fin de condensar el solvente para evitar la emisión de gases, para que el agua alcance la temperatura de ebullición del solvente se añadió glicerina. Con este experimento los autores encontraron que el Tolueno fue el solvente con mejores resultados, eliminando el 95\% de EVA después de 50 minutos a 60°C en baño ultrasónico \citep{Azeumo2019}.  

 Estos estudios de reciclaje de celdas no habían tomado en cuenta la producción de contaminantes durante el proceso, hasta que \citet{Fiandra2019}, combina dos procesos térmicos y utiliza cromatografía de gases para analizar y determinar la fracción molar de los principales gases emitidos, también se establecieron muestras de estas durante el tratamiento para verificar el rendimiento de la combustión con un análisis cuantitativo y cualitativo de las sustancias producidas.\citep{Fiandra2019}. Sin embargo, aunque en los métodos químicos se ha mencionado que el uso de solventes tiene un impacto en el ambiente, no se ha especificado la toxicidad de estos ni de los residuos generados en el proceso.

Debido a que el proceso de reciclaje presenta problemas ambientales, es importante buscar métodos que minimicen estos impactos y así se puedan crear regularizaciones para el manejo adecuado de este tipo de residuos. 

\section{Planteamiento del problema}
\label{sec:Planteamiento del problema}
Existen diversos procesos de reciclaje de celdas solares, estos están en dependencia del material final que se desea obtener: la celda solar completa o el silicio y metales contenidos en esta. Para todos los procesos de reciclaje se necesita primero eliminar los polímeros de la estructura del panel solar: EVA y Backsheet. Este es el proceso que presenta mayor problema en el reciclaje, ya que su eliminación requiere temperaturas  mayores a 450$^{\circ}$C y ácidos o solventes a altas concentraciones durante largos periodos de tiempo, lo que representa un alto gasto de energía. 

Cuando se desea obtener el silicio y metales de la celda,  los solventes a altas concentraciones no representan un problema, pero si lo que se desea es obtener la celda completa para su reutilización, el uso de estos solventes y altas temperaturas puede presentar defectos en ella, por lo que no es considerado un método viable \citep{Kim2012}.

Los tratamientos actuales de residuos fotovoltaicos deben de tomar en cuenta la reducción de producción de sustancias tóxicas durante la recuperación de materiales, así como tener nuevas regularizaciones en el manejo de los productos peligrosos derivados del reciclaje \citep{Fiandra2019}. 

\section{Justificación}
\label{sec:Justificación}
El tiempo de vida útil de un panel solar es entre 20 y 30 años, perdiendo más del 25\% de su capacidad después de los 25 años. Después de este tiempo se convierten en desecho electrónico, cuyo manejo es difícil a causa de su toxicidad creciente \citep{Doi2001}. Se han desarrollado procesos de reciclaje de paneles que permiten reutilizar los materiales para crear nuevos paneles u otros equipos. Para cualquier método de reciclaje utilizado es indispensable primero eliminar  EVA y Backsheet del módulo fotovoltaico, que al ser degradados térmicamente emiten gran cantidad de gases tóxicos y eliminados por procesos químicos se utilizan solventes ácidos que, de no ser manejados adecuadamente, provocan un grave impacto ambiental \citep{Fiandra2019}.

El método térmico es uno de los más estudiados, sin embargo debido a la alta cantidad de gases tóxicos que emite, lo hace peligroso para la salud de quien lo realiza y el ambiente. En comparación, el método químico deja residuos líquidos de solventes que con un manejo adecuado evitan impactos en la salud y el medio ambiente. Al realizar un estudio de este método dará posibles soluciones a los impactos ambientales que genera y optimizará el proceso de reciclaje de paneles solares haciéndolos más amigables con el medio ambiente. 

\section{Objetivo general}
\label{sec:Objetivo general}
Evaluar el método químico para eliminar el EVA de un panel solar tomando en cuenta la eficiencia e impacto ambiental

\section{Objetivos específicos}
\label{sec:Objetivos específicos}
1.	Evaluar la eficiencia del tratamiento químico, utilizando Hexano, Benceno y Tolueno como solvente en la eliminación del EVA y su impacto al medio ambiente.

2.	Optimizar el proceso químico con el solvente de mejor resultado. 

3.	Analizar la toxicidad de los residuos de los solventes utilizados. 


