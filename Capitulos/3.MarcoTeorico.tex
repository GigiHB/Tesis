\section{Efecto fotovoltaico y materiales semiconductores}
\label{sec:efectofoto}

La solar fotovoltaica es un tipo de energía renovable, convierte la luz del sol en energía eléctrica mediante el efecto fotovoltaico, en el que los fotones de la luz solar se convierten en energía eléctrica. Para aprovechar este efecto, se ensamblan las celdas solares en una estructura denominada módulo fotovoltaico, en la cual se conectan eléctricamente y se encapsulan para protegerlas del ambiente. Estas celdas están hechas de materiales semiconductores, cuya conductividad eléctrica es inferior a la de los metales, pero mayor a la de los aislantes. Los semiconductores tienen incompleto su orbital externo, que usualmente tiene 4 electrones de valencia (como en el caso del silicio), Estos electrones de valencia forman una red cristalina uniéndose con 4 átomos vecinos mediante enlaces covalentes. 

Cuando a un material semiconductor se le aporta la suficiente energía, como una elevación de temperatura o iluminación, los electrones de valencia pueden romper el enlace y moverse a través de la red cristalina, dejando una vacante (hueco). La energía mínima para que esto suceda se conoce como de enlace, y es una constante que varía dependiendo del material semiconductor, por ejemplo la del silicio es de 1.12 eV.

La luz solar tiene distintas longitudes de onda que componen el espectro electromagnético solar, el espectro solar visible se sitúa entre 1.6 y 3.1 eV, por lo que cuando un fotón de luz solar incide en un material semiconductor le da la suficiente energía para romper el enlace y generar un electrón-hueco \citep{Mesa2009}. 

Es indispensable que los electrones y huecos se agrupen en distintas zonas del material para poder formar el campo eléctrico, esto se consigue formando una unión PN que separa los portadores de electrones y los huecos en una celda solar y genera un voltaje. Para formar esta unión se añaden impurezas tipo P por un lado del material y N por otro. La región tipo P tendrá una alta concentración de huecos y la N de electrones, cuando estos se mueven al otro lado de la unión dejan cargas expuestas que están fijos a la red cristalina. Del lado P los núcleos de iones negativos están expuestos, y del lado N los positivos, formando el campo eléctrico. 

Una variedad de materiales y procesos pueden satisfacer potencialmente los requisitos para la conversión de energía fotovoltaica, pero en la práctica casi toda la conversión de energía fotovoltaica utiliza materiales semiconductores en forma de unión PN. Las celdas solares son dispositivos que convierten la luz solar en electricidad. Este proceso requiere, en primer lugar, un material semiconductor en el que la absorción de luz eleva un electrón a un estado de mayor energía y, en segundo lugar, el movimiento de este electrón de mayor energía desde la celda solar a un circuito externo. El electrón luego disipa su energía en el circuito externo y regresa a la celda solar.   

\section{Paneles fotovoltaicos}
\label{sec:paneles}

 Los paneles solares se clasifican de acuerdo a la tecnología, existen tres generaciones de estas. La primera generación son los paneles solares de silicio (monocristalino y policristalno), la segunda generación son los panales solares elaborados con películas delgadas (silicio amorfo, teluro de cadmio, seleniuro de cobre, indio y galio o CIGS), y las de tercera generación son las tecnologías emergentes (fotovoltaica de concentración, paneles solares sensibilizados por colorantes, paneles solares orgánicos e híbridos)\citep{Xu2012}. En este manuscrito nos enfocaremos en los paneles solares de primera generación, específicamente aquellos basados en silicio cristalino, los cuales dominan actualmente el mercado fotovoltaico. Según un estudio publicado en la revista Joule, en 2023, los paneles de silicio cristalino representaron aproximadamente el 95\% del mercado global de paneles solares, y se espera que mantengan su liderazgo durante los próximos años debido a su eficiencia y durabilidad \citep{ITRPV2023}. 
 
 \subsection{Paneles de silicio monocristalino y policristalino}
 \label{sec:Silicio monocristalino y policristalino}
 
 Los paneles solares de silicio son la tecnología que actualmente domina la capacidad instalada mundial. Algunas de sus ventajas sobre las otras tecnologías de paneles son la abundancia de materia prima, el proceso de fabricación semejante a la de la tecnología microelectrónica o circuitos integrados, y sus propiedades eléctricas estables, lo que garantiza su funcionamiento por más de 25 años \citep{ITRPV2023, Yasuhiro2011}.   
 
 Las celdas solares de silicio monocristalino son crecidos mediante la técnica Czochralski, a partir de lingotes cilindricos de gran pureza. Se reconocen a simple vista ya que su superficie es uniforme (figura \ref{fig:mono y poli}). De acuerdo con el NREL, las celdas solares de silicio monocristalino tienen una eficiencia de 26.1\%.Los paneles solares de silicio policristalino alcanzan una eficiencia de hasta 23.3\%, ligeramente inferior a la de los paneles de silicio monocristalino. Esta diferencia se debe, en parte, al menor grado de pureza del silicio utilizado. La producción de silicio policristalino comienza con la reducción del cuarzo (SiO$_{2}$) que reacciona con carbono para obtener silicio metalúrgico con una pureza del 98–99\%. Este material se somete posteriormente a un proceso de purificación, comúnmente el método Siemens, mediante el cual se alcanza una pureza de grado solar ($\geq$99.9999\%). Una vez purificado, el silicio se funde a una temperatura cercana a los 1414°C y se vierte en moldes de grafito o cuarzo. Durante la solidificación controlada, se forman múltiples cristales (granos) con diferentes orientaciones, lo que caracteriza la estructura policristalina. \citep{Funcan2011, Aully2016}.
  
 \begin{figure}[htb]
 	\begin{center}
 		\includegraphics[width=0.5\textwidth]{./Figuras/monoypoli.png}
 	\end{center}
 	\vspace{-1em} %Para reducir el espacio antes del caption
 	\caption{Celda solar de silicio monocristalino y policristalino}
 	\label{fig:mono y poli}
 \end{figure}
 
  \subsection{Estructura de paneles solares de silicio}
 \label{sec:Estructura paneles}
  
  Un panel solar de silicio tiene la estructura presentada en la figura \ref{fig:Componentes panel}. El marco de aluminio representa el 25\% del peso del panel, y da resistencia a la estructura. La cubierta frontal, usualmente es de vidrio, que representa el 85\% del panel y es lo que protege a la celda solar de los factores atmosféricos. El encapsulante, es de etil vinil acetato (EVA), representa del 5 al 10\% del peso del panel, protege a las conexiones de la celda solar y a la vez es el aditivo que une la celda con el vidrio y la cubierta posterior. Esta última está elaborada con una capa de polivinil fluorado (PVF, por sus siglas en inglés) comercialmente registrado por la marca DuPont como Tedlar, es lo que da soporte mecánico al panel solar y representa el 1.5\% del peso de este \citep{Fiandra2019}. 
  
  \begin{figure}[htb]
  	\begin{center}
  		\includegraphics[width=0.5\textwidth]{./Figuras/Componentes.jpg}
  	\end{center}
  	\vspace{-1em} %Para reducir el espacio antes del caption
  	\caption{Composición típica de un panel solar}
  	\label{fig:Componentes panel}
  \end{figure}
  
\section{Características de los componentes del panel solar}
\label{sec:caracteristicas de componentes}  

\subsection{Cubierta frontal}
\label{sec:cubierta frontal}   

 La cubierta frontal protege a la celda solar de los factores atmosféricos e impactos. Para que los rayos solares puedan penetrar esta cubierta y llegar a la celda, el material tiene que ser trasparente, por lo que usualmente se utiliza vidrio, que permite el paso del 91.5\% de la luz que incide. Está compuesto de SiO$_{2}$ (70\%), Na$_{2}$O (15\%), CaO (10\%), y se agrega entre 1 y 4\% de MgO para prevenir desvitrificación y 0.5\% de Al$_{2}$O$_{3}$ para aumentar su duración.  
 
 \subsection{Cubierta posterior}
 \label{sec:cubierta posterior} 
 
 Conocido como backsheet, es una multicapa de varios polímeros, el más usado es el polivinil fluorado (comercialmente llamado Tedlar). El backsheet protege al módulo de la humedad y otros agentes, además de que lo aísla eléctricamente. Es habitual que sea color blanco para reflejar la luz que no recogen las celdas sobre la cara posterior \citep{Kim2012}.
 
 Algunos de los polímeros y termoplásticos usados como Backsheet son:\\
 •	Fluoruro de polivinilo (PVF), registrado por DuPon como Tedlar\\
 •	Tereftalato de polietileno (PET)\\
 •	Poliamida (PA)\\
 •	Prolipropileno (PP)\\
 •	Fluoruro de polivinilideno (PVDF)\\
 •	Etileno-clorotrifluoroetileno (ECTFE)\\
 •	Terpolímero de tetrafluoroetileno, hexafluoropropileno y fluoruro de vinilideno (THV)\\ 
 
 Los Backsheets comerciales se muestran en la tabla \ref{tab:BScomerciales}. Los basados en fluoruro de polivinilo son: FT (tedlar y poliester) y TPT (tedlar-poliester-tedlar).
 
 \begin{table}[htb]
 	\caption{Backsheets comerciales \citep{Klaus2016}.}
 	\vspace{-0.5em} %Para reducir el espacio después del caption
 	\label{tab:BScomerciales}
 	\begin{center}
 		\begin{tabular}{c||c|c}\hline
 			\textbf{Nomenclatura} & \textbf{Fabricante} & \textbf{Nombre comercial} \\ \hline
 			\textbf{TPT-1} & Isovoltaic & Icosolar FPF 2442 \\ \hline
 			\textbf{TPT-2} & Krempel & PTL 3 – 38/250 \\ \hline
 			\textbf{TPT-3} & Krempel & PTL 3 HR 1000V \\ \hline
 			\textbf{FP-1} & Isovoltaic & Icosolar FPA 3572 \\ \hline
 			\textbf{FP-2} & Krempel & PVL Akasol 1000 V \\ \hline
 			\textbf{FP-3} & Honeywell & Powershield PV 270 \\ \hline
 			\textbf{FP-4} & 3M & Scotchshield 17T \\ \hline
 			\textbf{PET-1} & Dunmore &  Dun Solar PPEþ1360 \\ \hline
 			\textbf{PET-2} & Coveme &   DyMat PYE 3000 \\ \hline
 			\textbf{PET-3} & Isovoltaic &    Icosolar APA 3552 \\ \hline
 			\textbf{PA-1} & Isovoltaic &    Icosolar AAA 3554 \\ \hline      
 		\end{tabular}
 	\end{center}
 \end{table} 

\citet{Klaus2016}, obtuvo algunas caracterizaras de los backsheet comerciales más utilizados, que se muestran en la tabla \ref{tab:BSPropiedades}. Las propiedades fueron ponderadas al espectro solar AM 1.5, con reflectancia (R) solar hemisférica 97\% difusa y 2.9\% directa, transmitancia (T) 95.7\% difusa y 4.3\% directa. Las propiedades mecánicas fueron obtenidas por pruebas de tracción monótona, el módulo de Young (E) indica la relación existente entre los incrementos de tensión y deformación longitudial (elasticidad), mientras que $\sigma_{break}$ representa los valores de estrés en la rotura del Backsheet en MPa.  

\begin{table}[htb]
	\caption{Propiedades ópticas y mecánicas de distintos Backsheets \citep{Klaus2016}.}
	\vspace{-0.5em} %Para reducir el espacio después del caption
	\label{tab:BSPropiedades}
	\begin{center}
		\begin{tabular}{c||c|c|c|c}\hline
			\textbf{Nomenclatura} & \textbf{R} & \textbf{T} & \textbf{E (MPa)} & \textbf{$\sigma_{break}$ (MPa)} \\ \hline
			\textbf{TPT-1} & 0.695 $\pm$ 0.004 & 0.017 & 2763 $\pm$ 18 & 132 $\pm$ 8 \\ \hline
			\textbf{TPT-2} & 0.695 $\pm$ 0.001 & 0.015 & 2701 $\pm$ 32 & 136 $\pm$ 6 \\ \hline
			\textbf{TPT-3} & 0.793 $\pm$ 0.004 & 0.102 & 3148 $\pm$ 5 & 146 $\pm$ 5 \\ \hline
			\textbf{FP-1} & 0.789 $\pm$ 0.002 & 0.153 & 2738 $\pm$ 18 & 129 $\pm$ 13 \\ \hline
			\textbf{FP-2} & 0.769 & 0.167 & 2763 $\pm$ 29 & 118 $\pm$ 13 \\ \hline
			\textbf{FP-3} & 0.780 $\pm$ 0.006 & 0.160 $\pm$ 0.001 & 2038 $\pm$ 19 & 89 $\pm$ 5 \\ \hline
			\textbf{FP-4} & 0.762 $\pm$ 0.004 & 0.180 $\pm$ 0.001 & 770 $\pm$ 8 & 30 $\pm$ 1 \\ \hline
			\textbf{PET-1} & 0.702 $\pm$ 0.008 & 0.219 & 2195 $\pm$ 27 & 86 $\pm$ 1  \\ \hline
			\textbf{PET-2} & 0.798 $\pm$ 0.025 & 0.154 $\pm$ 0.008 & 2233 $\pm$ 19 & 112 $\pm$ 9 \\ \hline
			\textbf{PET-3} & 0.811 $\pm$ 0.001 & 127 & 2517 $\pm$ 93 & 123 $\pm$ 13 \\ \hline
			\textbf{PA-1} & 0.838 & 0.086 $\pm$ 0.001 & 692 $\pm$ 15 & 31 $\pm$ 1 \\ \hline      
		\end{tabular}
	\end{center}
\end{table} 

 \subsection{Encapsulante}
\label{sec:encapsulante} 

El Etil Vinil Acetato es el polímero utilizado para encapsular la celda solar, actua como termoplástico y elastómetro dependiendo de la cantidad de vinil acetato (VA). Para aplicaciones en paneles solares el EVA contiene alredor de un 28-33 \%wt de vinil acetato, compuesto con aditivos como agentes de curado (tabla \ref{tab:AdhitivosEVA}), absorbedores UV, foto-oxidantes y termo-oxidantes \citep{Badiee2016}. 

\begin{table}[htb]
	\caption{Componentes aditivos del EVA para paneles fotovoltaicos \citep{Olivera2017}.}
	\vspace{-0.5em} %Para reducir el espacio después del caption
	\label{tab:AdhitivosEVA}
	\begin{center}
		\begin{tabular}{|p{4cm}||p{4cm}|p{5cm}|}\hline
			\textbf{Componente} & \textbf{\%(w/w) para EVA 96-98\%} & \textbf{Propiedad}\\ \hline
			\textbf{Peróxido} & 1 a 2 & Agente de curado, se utiliza para reticular a temperaturas elevadas durante laminación \\ \hline
			\textbf{Benzotriazol} & 0.2 a 0.35 & Absorbedor UV\\ \hline
			\textbf{Hinder estabilizador de luz de amina (HALS)} & 0.1 a 0.2 & Estabilizador UV, antioxidante primario, descompone los radicales de peróxido\\ \hline
			\textbf{Fosfonito fenólico} & 0 a 0.2 & Antioxidante secundario, descompone el peróxido, elimina los radicales\\ \hline
			\textbf{Trialcoxisilanos} & 0.2 a 1 & Promueve la adhesión del EVA con superficies inorgánicas\\ \hline	
		\end{tabular}
	\end{center}
\end{table} 

Una medición crítica para la calidad de adhesión del EVA es el contenido de gel. Un contenido de gel arriba del 70\% se considera aduecuado, este se mide con analísis químico o prueba de "peeling". Mientras mayor sea el porcentaje de contenido de gel, mejor será la adhesión del EVA, un contenido de gel del 100\% implica que el polímero se ha incorporado completamente en el reticulado de la estructura\citep{Olivera2017}.

Algunas de las propiedades del EVA utilizado en paneles fotovoltaicos son: \\
•	Alta resistividad eléctrica\\ 
•	Alta resistividad de volumen (0.2–1.4×10$^{16}$ $\Omega$-cm) \\
•	Temperatura de reticulación relativamente baja\\
•	Resistencia a la radiación UV\\
•   Baja absorción / entrada de agua\\
•	Contenido de gel superior al 70\% después del curado\\
•	Alta transmisión óptica (superior al 91\%)\\
•	Alta adhesión al vidrio (a 90$^{\circ}$C fuerza de pelado de 9-12 Nmm$^{-1}$)\\

 \subsection{Celda solar}
\label{sec:Celda solar}

Los paneles de primera generación utilizan celdas solares basadas en silicio como material semiconductor. Estas celdas están formadas por dos capas principales: una capa tipo P, ubicada en la parte inferior y dopada con átomos de boro, los cuales actúan como aceptores de electrones; y una capa tipo N, situada en la parte superior y dopada con átomos de fósforo, que funcionan como donadores de electrones. La unión entre ambas capas, conocida como unión PN, genera un campo eléctrico interno que facilita la separación de cargas y permite la generación de corriente eléctrica al incidir la luz solar sobre la celda. \citep{PVeducation2021}. 

Para recolectar adecuadamente los portadores de carga tienen un enrejado metálico, los electrodos o contactos eléctricos deben tener bajos valores de resistencia en serie y buena adhesión mecánica, además deben ser soldables para su interconexión. El área total ocupada por los electrodos frontales debe de ser del 5 al 7\%, siendo esta la cantidad de luz incidente perdida. Para los electrodos posteriores usualmente se emplean toda su área, pero en ocasiones se colocan en forma de reja (figura \ref{fig:rejilla}) para reducir el efecto de recombinación superficial.

    \begin{figure}[htb]
  	\begin{center}
  		\includegraphics[width=0.5\textwidth]{./Figuras/rejilla.png}
  	\end{center}
  	\vspace{-1em} %Para reducir el espacio antes del caption
  	\caption{Estructura de celda solar con electrodos posteriores en rejilla \citep{Yasuhiro2011}}
  	\label{fig:rejilla}
  \end{figure}
  
Debido a que el indice reflexión del silicio varia entre 6 a 3.5 en el rango espectral de 400 a 1100 nm, existe una perdida debido al reflejo espectral de 54\% en ondas cortas y 34\% en longitudes de onda largas. Con la finalidad de reducir el reflejo, se utiliza una capa antireflejante en la celda (figura \ref{fig:rejilla}). Aun con esta capa antireflejante, ya que las celdas solares tienen una superficie lisa, pueden existir perdidas por reflexión, por lo que algunas celdas se texturizan con forma piramidal, como se observa en la figura. Con esta textura superficial en la oblea de silicio con orientación <100>, la luz incidente se refleja, pero esta se dirige a la superficie piramidal vecina <111>, provocando una reflexión múltiple \citep{Yasuhiro2011}. 

  \begin{figure}[htb]
	\begin{center}
		\includegraphics[width=0.5\textwidth]{./Figuras/texturacelda.png}
	\end{center}
	\vspace{-1em} %Para reducir el espacio antes del caption
	\caption{Textura piramidal de la superficie de la celda \citep{Yasuhiro2011}}
	\label{fig:textura}
\end{figure}

 \subsection{Caja de conexiones}
\label{sec:caja de conexiones}

Esta colocada en la cara posterior del panel solar, cuenta con bornes de conexión negativo y positivo, además de diodos de paso de cobre. Tiene una protección recomendada contra el polvo y agua, es fabricada con plásticos resistentes a altas temperaturas. 

\section{Fin de vida de paneles fotovoltaicos}
\label{sec:fin de vida}

El principio de reducción de basura se basa en las 3R´s: Reducir, Reusar y Reciclar, es de los preferidos al momento de hablar del fin de vida de los paneles fotovoltaicos. Lo primero que se debe tomar en cuenta es tratar de reducir el volumen de basura generada, después el reutilizar para así evitar generar más residuos y finalmente el reciclar los que no se puedan evitar generar. 

 \subsection{Reducir}
\label{sec:reducir}
Actualmente las nuevas tecnología de paneles fotovoltaicos toman en cuenta la eficiencia de recursos y materiales, esto significa utilizar los recursos limitados de manera sostenible, minimizando el impacto sobre el ambiente. Si bien la composición de paneles solares de primera generación no ha cambiado significativamente, se ha logrado un ahorro de material considerable, esto gracias a que las recientes investigaciones se han centrado en minimizar la cantidad de materiales por panel, principalmente con el fin de disminuir su costo, dado que el consumo de materiales y metales raros y valiosos aumentará a medida que crezca el mercado fotovoltaico, la disponibilidad y los precios impulsaran los esfuerzos de reducción y sustitución. Algunas de las investigaciones se enfocan en la reducción de:

\textbf{Vidrio}\\
Una mayor optimización de la composición del vidrio, el espesor, el revestimiento antirreflectante y las estructuras de la superficie aumentará la transmisión de los cristales frontales en otro 2\% para 2024. El uso de vidrio podría ser de dos milímetros de espesor o incluso menos sin embargo esto requerirá más estabilización mecánica. 

\textbf{EVA y backsheet}\\
 Los polímeros no se reciclan porque los materiales termoestables que dominan el mercado, como los utilizados en encapsulantes (EVA) y backsheet (tedlar), no se pueden disolver o fundir para reciclar sin descomponerse. Esto impide su reprocesamiento y reutilización, siendo una limitante en el reciclaje de paneles solares. Por ello investigaciones buscan reducir o reemplazar la cantidad de polímeros, especialmente para los backsheet, y desarrollar soluciones innovadoras que permitan mejorar la circularidad de estos materiales. Entre las estrategias propuestas destacan el diseño de encapsulantes termoplásticos o con enlaces químicos reversibles, el uso de estructuras modulares que faciliten la separación mecánica de capas, y el avance de procesos químicos selectivos para la recuperación de materiales sin degradación significativa. Estas iniciativas buscan simplificar el desensamble, mejorar la calidad del material recuperado y reducir los costos asociados al reciclaje \citep{Kang2021, Chen2023, IEAPVPS2020}


\textbf{Silicio}\\
Se busca que las celdas solares sean más delgadas para reducir la cantidad de silicio utilizado. Por ejemplo, al pasar a un diseño de celda de contacto posterior, el uso de silicio podría reducirse a la mitad y el consumo de energía podría reducirse en aproximadamente un 30\%.

\textbf{Plata}\\
Aproximadamente el 95\% de las celdas solares de c-Si se producen ahora con líneas de contactos de plata serigrafiado en la parte frontal que cubren aproximadamente entre el 6\% y 8\% del área de la celda. Se espera una reducción significativa de la plata, según el estudio International Technology Roadmap for Photovoltaic (ITRPV), es posible el uso de otros metales como el cobre en combinación con níquel y aluminio \citep{Irena2016}.

 \subsection{Reusar}
\label{sec:reusar}
 
Las fallas de un panel que pueden surgir antes del final de su vida son una oportunidad de reparación o reúso. Si se descubren defectos durante la fase inicial de la vida útil de un panel fotovoltaico, los clientes pueden intentar reclamar garantías de reparación o reemplazo, siempre que el socio contratante aún exista. En algunos países existen compañías de seguros que pueden estar involucradas para compensar algunos o todos los costos de reparación o reemplazo dentro de los acuerdos contractuales. En tales casos, la propiedad de los paneles a menudo cambia a la compañía de seguros. Para recuperar algo de valor de un panel devuelto mediante la reventa, se deben realizar pruebas de calidad que verifiquen principalmente la seguridad eléctrica y la potencia de salida. Una caracterización de prueba de flash y una prueba de fuga húmeda es un ejemplo. Cuando las reparaciones son necesarias y factibles, normalmente implican la aplicación de un nuevo marco, una nueva caja de conexiones, reemplazo de diodos, nuevos enchufes etc.

Los paneles fotovoltaicos reparados se pueden revender como reemplazos o como paneles usados a un precio de aproximadamente el 70\% menor al precio de venta original. Ya ha surgido un mercado de paneles usados apoyado por plataformas virtuales de Internet como www.secondsol.de y www.pvXchange.com. Con cada vez más fotovoltaica instalada, el número de estos paneles o componentes de segunda mano puede aumentar, generando un mercado para su reuso.

 \subsection{Reciclar}
\label{sec:reciclar}

Las tecnologías de reciclaje han sido estudiadas durante los últimos años. Este conocimiento ha proporcionado una base para el desarrollo de plantas de reciclaje especializadas, una vez que los flujos de residuos son lo suficientemente grandes para una operación rentable. Empresas de energía solar fotovoltaica llevaron a cabo una amplia investigación, como AEG, BP Solar, First Solar, Pilkington, Sharp Solar, Siemens Solar, Solar International y muchas otras.

Los componentes principales de los paneles de c-Si, incluidos el vidrio, el aluminio y el cobre, se pueden recuperar mediante una separación puramente mecánica. Sin embargo, sin una combinación de pasos térmicos, químicos o metalúrgicos, los niveles de impurezas de los materiales recuperados podrían ser lo suficientemente altos como para reducir los precios de reventa. Por lo que usualmente los procesos de reciclaje combinan más de un método (IRENA e IEA, 2016).

La separación de los componentes principales, como el vidrio laminado, los marcos metálicos, el cableado y los polímeros, es el primer paso en los procesos de reciclaje para paneles de primera generación. Del marco puede ser recuperado aluminio por procesos metalúrgicos, aquí también pueden aparecer otros elementos como el silicio y níquel, que son
componentes típicos de aleaciones de aluminio y están presentes en muy pequeñas cantidades \citep{Chow2020}.

Después del desmantelamiento físico del marco de aluminio y la caja de conexiones del panel fotovoltaico, los componentes restantes son el vidrio, la celda solar encapsulada con EVA y el Backsheet. Existen varios métodos para separar el vidrio, algunas técnicas utilizan procesos físicos en los que trituran vidrio, celda y encapsulante para colocar los fragmentos en una solución y separar los componentes por diferencia de densidad, luego de esto aplican procesos químicos con el fin de eliminar impurezas \citep{Azeumo2019}. Otros autores han probado los pre-tratamientos térmicos para retirar el vidrio, usualmente estos no pasan los 300$^{\circ}$C, con esto el vidrio puede ser retirado manual o mecánicamente de una manera sencilla \citep{Pagnanelli2019} \citep{Fiandra2019}.  

Para eliminar los polímeros no reciclables, se utilizan dos procesos: químico y térmico, ya sea por separado o combinando ambos. El proceso térmico es el más comúnmente usado debido a la facilidad y rapidez para degradar los polímeros, y obteniendo el silicio y metales sin residuos de estos. Para el caso del backsheet, su degradación termica empieza a los 250$^{\circ}$C, en los que se rompe la cadena polímerica, y cuando alcanza los 330$^{\circ}$C empieza a formar subproductos fluorados que requieren sistemas de reducción. Cuando alzanza os 800°C se degrada completamente y emite CO$_{2}$, CO y compuestos orgánicos volatiles (VOC). El EVA se degrada completamente a los 500°C, la primera etapa ocurre a los 300$^{\circ}$C donde se elimina el ácido acético, dejando polieno como residuo. La segunda etapa de su degradación ocurre a los 400 y 500$^{\circ}$C donde se descompone la cadena restante de la primera etapa. La degradación térmica de ambos polímeros emite gases tóxicos, aunque en el caso del EVA son en menor cantidad, por lo que muchas técnicas de reciclaje prefieren retirar mecánicamente el backsheet y luego llevar a un tratamiento térmico la celda para eliminar el encapsulante \citep{Fiandra2019}. 

En los métodos químicos para la separación del vidrio y la eliminación del EVA y Backsheet, se sumerge el panel, previamente desmantelado, en ácidos o solventes orgánicos. El uso de estos solventes implica un manejo de los residuos que muchas veces no es llevado a cabo, lo que conlleva a un fuerte impacto en el ambiente. Por esta razón usualmente se acompaña el tratamiento químico con un pre-tratamiento térmico, en el que calientan el Backsheet a medida que se ablande para poder despegarlo manualmente \citep{Azeumo2019}, \citep{Pagnanelli2019}. Así solo se sumerge el vidrio y la celda encapsulada con EVA, usando menor concentración de solventes. Existen diferentes solventes orgánicos estudiados para la eliminación del EVA de la celda solar, como se muestra en la tabla \ref{tab:efecto de sol}, que indica el tipo de solvente usado y el efecto que tiene en el EVA después de diferentes tiempos en tratamiento \citep{Sukmin2007}. 

\begin{table}[htb]
	\caption{Efecto en la disolución del EVA de distintos solventes \citep{Sukmin2007}. }
	\vspace{-0.5em} %Para reducir el espacio después del caption
	\label{tab:efecto de sol}
	\begin{center}
		\begin{tabular}{c||c|c|c|c|c|c}\hline
			\textbf{Solvente} & \textbf{1min} & \textbf{30min} & \textbf{120min} & \textbf{210min} & \textbf{24hr} & \textbf{48hr} \\ \hline
			\textbf{2-Propanol} & x & x & x & x & x & x  \\ 
			\textbf{4-Metil-2-Pentanona} & H & H & H & H & H & H \\
			\textbf{Benceno de Petróleo} & H & H & H & H & H & H \\ 
			\textbf{Tetrahidrofurano} & S,H & D & D & D & D & D \\	
			\textbf{Tricloroetileno} & D & D & D & D & D & D \\
			\textbf{Tolueno} & D & D & D & D & D & D \\
			\textbf{Diclorobenceno} & H & D & D & D & D & D \\
			\textbf{Glicerina} & x & x & x & x & x & x  \\
			\textbf{Acetona} & x & x & x & x & x & x  \\
			\textbf{Alcohol etílico} & x & x & x & x & x & x  \\ \hline
			\multicolumn{7}{c}{\textbf{D}:Disuelto, \textbf{S}:Separado, \textbf{H}:Hinchado, \textbf{X}:Sin cambios } \\ \hline
		\end{tabular}
	\end{center}
\end{table}
 
La recuperación de pequeñas cantidades de materiales valiosos (plata, cobre), escasos (indio, telurio) o la mayoría de los materiales peligrosos (cadmio, plomo, selenio) como componentes puede requerir procesos adicionales y más avanzados. Estos se encuentran predominantemente en las fracciones de vidrio y encapsulante (polímero). El silicio puede ser recuperado por pirolisis, el programa japonés de Organización para el Desarrollo de Nuevas Energías y Tecnologías Industriales (NEDO, por sus siglas en inglés) creo una planta piloto que se basa en la pirólisis de los polímeros en un horno transportador, y posteriormente la recuperación del material semiconductor \citep{Irena2016}.

\textbf{Reciclaje de Vidrio}\\
El vidrio obtenido del reciclaje de paneles fotovoltaicos puede contener una capa de EVA, la cantidad restante de este polímero en el vidrio dependerá del método de separación usado y la eficiencia de este. Sin embargo, la presencia del EVA en el vidrio obtenido no impide su reciclaje dado que el EVA se puede descomponer en CO y CO$_{2}$ a 500$^{\circ}$C \citep{Fiandra2019}, estando por debajo de la temperatura de fusión del vidrio templado próxima a los 1500-1600$^{\circ}$C \citep{Herrarte2020,RaKim2017}. 

Para el reciclaje del vidrio se siguen las siguientes etapas: \\
1.- Triturado: El vidrio obtenido a partir de paneles solares es triturado, los trozos no deben exceder de 25mm de diámetro. \\
2.- Tamizado: Los trozos de vidrio pasan por una cámara de soplado con el fin de retener el polvo de vidrio en filtros, esto para evitar que se volaticen junto con los gases de combustión y minimizar el impacto ambiental.\\
3.- Lavado: El polvo remanente se termina de extraer llevando el vidrio a un lavado con agua caliente y agitación. \\
4.-Secado: Después de decantar el polvo de vidrio en el lavado, se procede a secar.\\ 
5.-Fundición: El vidrio pasa a fundirse en un horno con 20\% de vidrio virgen.

El vidrio utilizado en paneles fotovoltaicos es de baja impureza, por lo que su reciclaje podría usarse 100\% como materia prima. Sin embargo, ensayos han demostrado que su uso en cantidades mayores al 80\% provocan un incremento en su fragilidad, es por eso que se funde con 20\% de vidrio virgen con el fin de obtener material de calidad \citep{RaKim2017}.      

\textbf{Reciclaje de silicio}\\ 
Son muy pocos los trabajos que explican el reciclaje del silicio obtenido de paneles fotovoltaicos desechados. El proceso de reciclaje descrito a continuación se aplica para cualquier fuente de silicio \citep{Louis2018}. 
El silicio utilizado en la industria para fabricación de metales o aleaciones de aluminio cuenta con una pureza del 99\%, este se conoce como silicio de grado metalúrgico. El que es utilizado en la industria de microelectrónica o fotovoltaica debe ser purificado a tal grado de tener impurezas de 0.001 p.p.m. es decir, menos de un átomo de impureza por cada millón de átomos de silicio \citep{Martil2016}. Debido a que la pureza es la clave para el uso que se le dará al silicio, el proceso de reciclaje consiste en una purificación después de ser obtenido de los paneles solares desechados. 

El silicio puede contener carburo de silicio, óxido de silicio e impurezas metálicas. Estás últimas pueden ser eliminadas fácilmente mediante un lavado con ácido fluorhídrico, aún así, la separación de partículas de carburo de silicio es más complicada debido a su tamaño y densidad relativamente similar al silicio \citep{Martil2016,Louis2018}. Para facilitar la separación del carburo de silicio \citet{DeSousa2014} propuso la sedimentación bajo un campo electríco. Con esto la pureza alcanzada es del 98\%.       
Otros estudios sugieren la eliminación del carburo de silicio mediante la cementación del aluminio bajo argón a 1,500$^{\circ}$C, a partir de una aleación de aluminio y silicio. Sin embargo, el silicio se contamina con carburo de aluminio (Al$_{4}$C$_{3}$) \citep{Louis2018}.

\textbf{Reciclaje de metales}\\
Los metales como el cobre y aluminio se encuentran en los contactos, a los cuales se les aplica pintura de plata para mejorar su conductividad eléctrica. Otros metales pueden estar presentes debido a la pintura antirreflejante, esta puede ser de nitruro de silicio (Si$_{3}$N$_{4}$), Óxido de titanio (TiO$_{2}$), Óxido de silicio (SiO$_{2}$), Óxido de aluminio (Al$_{2}$O$_{3}$), u Óxido tántalo (Ta$_{2}$O$_{5}$). 

Una digestión se utiliza para separar los metales de la celda solar para obtener el silicio. Al final de esta digestión se obtiene el silicio solido y una solución con los metales digeridos: Cu, Ag, Al, Pb \citep{Carlos2019,Klugman2010}. 

La plata comúnmente se recicla mediante una fusión pirometalúrgica tradicional. Los materiales se incineran para producir una ceniza que se mezcla con materiales que contienen plata. Esta mezcla se combina con fundentes y plomo, esta combinación produce una aleación de plata/plomo con pureza de 50\%. Para eliminar el plomo se lleva a un proceso de refinación de fuego adicional, donde el plomo se elimina por oxidación obteniendo plata de alta ley (98\%). Con un proceso químico adicional la plata puede obtener una pureza del 99.9\% \citep{Louis2018,Hillard2003}. 
 
\textbf{Reciclaje de EVA y backsheet}\\
Los polímeros del panel solar usualmente son eliminados por degradación. Los procesos térmicos para este proceso necesitan temperaturas de 500$^{\circ}$C (EVA) y 800$^{\circ}$C (backsheet) donde se emiten gases tóxicos en el proceso \citep{Fiandra2019,Farneth2012}. 
Debido a esto se han investigado procesos donde es posible recuperar los polímeros por métodos mecánicos y químicos para evitar el impacto ambiental del tratamiento térmico \citep{Azeumo2019,Fiandra2019,Pagnanelli2019}. 

A pesar de que el backsheet no puede reciclarse, el EVA es un termopolímero que se ablanda con calor y se endurece al enfriarse, gracias a esta característica este polímero es posible de reusar. 
El EVA se descompone termicamente en dos pasos. En el primer paso a 250$^{\circ}$C se descompone el acetato de vinilo y en el segundo paso, a temperaturas mayores, se descompone la cadena de hidrocarburos restante. Un estudio elaborado por \citet{Chitra2020} demostró que el EVA recuperado de paneles fotovoltaicos de silicio es termicamente estable a 250$^{\circ}$C y puede ser recuperado a 170$^{\circ}$C, sin ninguna emisión de gases. Pruebas de espectroscopia UV-visible muestran que el EVA recuperado de paneles fotovoltaicos es transparente de 500 nm a 950 nm, con algunas bandas de absorción en la región de UV bajo que pueden deberse a la absorción específica del polímero. Con propiedades similares a las del EVA comercial, es posible el uso de EVA reciclado en la industria fotovoltaica \citep{Chitra2020}.  

  