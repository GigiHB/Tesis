
\section{Introducción}
\label{sec:Introducción}
La energía fotovoltaica es una de las más populares dentro del sector energético renovable, la capacidad instalada ha crecido significativamente durante los últimos años. De acuerdo con la International Renewable Energy Agency (IRENA), la energía instalada en el mundo pasó de 222 GW en el 2015 a 709 GW en el 2020 \citep{IRENA2020}. Sin embargo, así como el mercado fotovoltaico crece, también lo hacen los residuos de estos, con una vida útil de 25 a 30 años, se espera que para el 2050 se tengan de 60 a 70 millones de toneladas de basura fotovoltaica (IRENA e IEA, 2016). 
 
 Las celdas solares de silicio cristalino dominan el mercado actual, con alrededor de 80\% del total de paneles solares instalados. Se espera que este material continúe como líder en el desarrollo de tecnologías fotovoltaicas por lo menos durante la siguiente década \citep{Tokoro2021}. Algunos de los componentes de los paneles solares de silicio son la celda solar, el encapsulante EVA (etil vinil acetato), vidrio, y el soporte posterior (backsheet). Otros elementos como plomo (Pb) y estaño (Sn) se utilizan en cintas y soldadura, mientras que el cobre (Cu) se ha utilizado generalmente en el cableado de las conexiones. Estos elementos constituyen una parte significativa de los desechos fotovoltaicos. De acuerdo con reportes basados en el Procedimiento de Lixiviación según la Característica de Toxicidad (TCLP, por sus siglas en inglés), los lixiviados provenientes de módulos solares rotos pueden contener concentraciones de plomo (Pb) de hasta 9.3 mg/L. En las pruebas de Lixiviación con Precipitación Sintética (SPLP), los niveles alcanzan hasta 1.4 mg/L, mientras que en los ensayos con pH estático se han reportado concentraciones de hasta 6.7 mg/L, superando los límites permisibles establecidos por dichas metodologías. Aunque el encapsulante y el vidrio limitan parcialmente la movilidad de los metales cuando el módulo permanece intacto, factores como el envejecimiento del panel, el pH del entorno y los daños físicos durante el manejo o la disposición final pueden favorecer la liberación de estos elementos peligrosos \citep{Bhakta2021}. 
 
 La IRENA, en su estudio de fin de vida de paneles solares, clasifica la basura fotovoltaica por tipo de pérdida: regular o temprana. Las pérdidas regulares son las de aquellos paneles solares con pérdida de eficiencia después de 25 años (vida útil), y las pérdidas tempranas son de aquellos que tuvieron algún defecto y fueron desmantelados antes del fin de su vida útil. De acuerdo con un estudio realizado por \citet{Chow2020}, el 19\% de los paneles solares se retiran por fallas en la caja de conexiones, el 10\% debido a defectos en la celda, 20\% por fallas ópticas, 19\% por fin de vida útil y el resto por problemas de instalación y transporte.
 
 La mayoría de los países clasifican la basura fotovoltaica como residuos industriales o generales. La Unión Europea (EU), los clasifica como desechos de equipos electrónicos y eléctricos (WEEE, por su siglas en inglés), con el código 160214 del catalogo europeo de residuos (EWC, por sus siglas en inglés)\citep{Azeumo2019}. Mientras que países como Estados Unidos y Japón, tienen regulaciones que incluyen detección y pruebas de materiales peligrosos, tratamiento, reciclaje o eliminación \citep{Xu2012}. Con estas regulaciones, se han creado procesos de reciclaje de módulos fotovoltaicos, IRENA e IEA (2016), evaluaron los métodos existentes, concluyendo que el reciclaje de alto valor (obtención y reutilización de los componentes con mayor beneficio económico) es la opción preferida de las tecnologías, lo cual solo asegura la recuperación de un porcentaje pequeño de todo el panel. Es por eso que la directiva de WEEE de la EU, ha establecido que el reciclaje de alto valor también debe incluir la obtención de elementos de alto valor energético, no solo económico, como el silicio y vidrio. A su vez, deben considerar la calidad del material recuperado y la eliminación de sustancias potencialmente dañinas. Sin embargo otros países, como México, aún no cuentan con regulaciones de desechos de panales fotovoltaicos y con el crecimiento de las instalaciones es necesario contar con métodos para el correcto manejo de los residuos al finalizar su vida útil. 
 
  
\section{Antecedentes}
\label{sec:Antecedentes}

El reciclaje de paneles fotovoltaicos ha sido investigado a través de los años, se han creado distintos métodos en dependencia del material final a reciclar, a la fecha los dos más usados por las empresas son los hidrometalúrgicos y térmicos. El primer proceso es más complejo debido a las operaciones unitarias que requiere, sin embargo, los térmicos, a pesar de que cuentan con un diseño más simple, tienen más problemas ambientales además de que consumen más energía \citep{Ardente2019}.

Solar World utilizó tratamientos térmicos en el reciclaje de paneles solares en el 2003. Para separar las capas del panel usó temperaturas de 600°C, con lo que logró eliminar el Etil Vinil Acetato (EVA), el polímero que encapsula la celda y la une al vidrio y backsheet. Posteriormente, la celda solar se trata con ácido para separar los metales \citep{appropedia_pv_recycling}. Debido a la diversidad de materiales presentes en los módulos fotovoltaicos, es común que las empresas combinen múltiples procesos con el objetivo de maximizar la recuperación de componentes valiosos. Un ejemplo de ello es la tecnología Sunicol, que integra un tratamiento térmico con uno de grabado químico. En la etapa térmica, se separan los plásticos y se desmontan el vidrio y el marco de aluminio para su posterior reciclaje; en la etapa de grabado, se disuelve el EVA, permitiendo recuperar el silicio \citep{Irena2016}.

Un proceso similar fue desarrollado en Japón mediante el programa NEDO, a través de la iniciativa FAIS, en donde se emplea también la pirólisis de los polímeros en un horno de cinta transportadora. A diferencia de otros sistemas, este procedimiento contempla la remoción previa del marco y de la lámina posterior (backsheet), lo que facilita la posterior recuperación del material semiconductor (silicio o CIS) y del vidrio reciclado \citep{Komoto2014}.

\citet{Sukmin2007} utiliza un método físico para separar el vidrio y backsheet de la celda, posteriormente, los trozos de celda con EVA son llevados a un proceso químico para la eliminación de este polímero, para después llevarlos a un tratamiento térmico con el fin de eliminar partículas de EVA restantes. Para el método químico probó distintos solventes: 2-propanol; tricloroetileno; o-diclorobenceno; acetona; alcohol etílico y tolueno. El proceso consistió en llevar trozos de celda con solvente durante distintos tiempos a 90°C. Los resultados fueron favorables desde el primer minuto con tolueno, el cual se reporta que disuelve el EVA, al igual que el diclorobenceno después de 30 min. Mientras que el alcohol etílico y acetona no tuvieron efecto alguno en el EVA. 

Para separar el vidrio, backsheet y eliminar el EVA, los procesos térmicos y mecánicos muchas veces rompe la celda solar, lo que hace imposible la recuperación de esta para su reutilización. Si las celdas solares pueden recuperarse sin romperse, estas son usadas directamente en nuevos paneles solares, esto es considerada la mejor opción de reciclaje, ya que elimina múltiples procesos que implica la recuperación del silicio para formar una nueva celda. Para recuperar la celda solar completa se necesita quitar el EVA que la encapsula, el único método con el cual se puede lograr esto sin romperla es el químico. Usar solventes para eliminar el EVA puede ser un proceso largo, como lo demostró \citet{Sukmin2007}, como se mencionó anteriormente, el proceso empleado para eliminar el EVA llevó tiempos de hasta dos días. \citet{Kim2012} utilizan un método químico para la recuperación de la celda entera, en el que para disminuir el tiempo de eliminación de EVA, crean un sistema con baño ultrasónico. El sistema cuenta con recirculación del solvente, utilizando tolueno, tricloetileno y diclobenceno a temperaturas de 55 y 70$^{\circ}$C y potencia ultrasónica de 450 y 900W durante 60 minutos. Con el proceso planteado observaron que se obtenía mayor radio de EVA disuelto con temperaturas de 70$^{\circ}$C y potencia de 900W, esto utilizando Tolueno y diclobenceno como solventes. No obstante, los autores reportaron que, en varios casos, el encapsulante se hinchaba durante la disolución, generando tensiones mecánicas que causaban fracturas en la celda. Aunque el silicio cristalino no reacciona directamente con los solventes, el proceso puede comprometer su integridad física, lo que limita seriamente la viabilidad del método. Por ello, este tipo de técnicas no son comúnmente implementadas en el reciclaje de paneles solares \citep{Kim2012}.

\citet{Azeumo2019}, prueba el método físico y químico para el reciclaje de paneles solares. Los módulos fotovoltaicos con el marco de aluminio y la caja de conexiones previamente desmantelados, fueron cortados en muestras de 13x13 cm, algunas de estas son llevadas a trituración, y el resto son puestas en tratamiento químico. El proceso físico consistió en fragmentar las muestras con un molino de cuchillas, para posteriormente ser pasadas a través de rejillas de 0.4cm, lo que pasaba la rejilla fue recolectado y los restos llevados de nuevo al molino hasta obtener el tamaño deseado. A estas muestras molidas se les añadió primero agua, posteriormente cloruro de sodio con densidad de 1.2g/cm$^3$, y por último politungstato de sodio con densidad de 1.5g/cm$^3$. Con la diferencia de densidad los polímeros flotan y la mezcla de metales y vidrio precipita, lo que facilita la recolección. A pesar de que la ruta física es más simple y práctica de realizar, la calidad de los metales es muy baja, debido a que estos aún contienen restos de EVA. El método químico consistió en poner la muestras con solvente en baño sonificado, probando 3 temperaturas: 25, 60 y 110$^{\circ}$C. Los solventes utilizados fueron heptano, tolueno y xileno. Los mejores resultados se mostraron con tolueno y xileno y que fueron éstos los que lograron disolver la mayor cantidad de EVA, eliminando el 95\% después de 60 min a 50$^{\circ}$C. La diferencia con el método evaluado por \citet{Kim2012}, en el que se logró el desprendimiento del 80\% del encapsulante utilizando tolueno y un baño sonificado a 900 $W$, radica principalmente en la concentración del solvente. El proceso realizado por Azeumo et al. emplea tolueno de alta pureza (99.9\%), mientras que \citet{Kim2012} utilizan una solución de tolueno diluido a 3M. En su estudio, concluyen que la concentración del solvente tiene un impacto más significativo en el desprendimiento del encapsulante de la celda que la potencia de sonificación. Sin embargo, ninguno de estos estudios aborda la toxicidad de los residuos generados por los solventes y el encapsulante, ni menciona si estos residuos requieren un tratamiento posterior o un manejo especial para mitigar sus impactos ambientales. 

Las tecnologías de reciclaje no solo deben enfocarse en la recuperación del silicio y metales, también deben tomar en cuenta sus demás componentes, \citet{Pagnanelli2019} prueba dos rutas para el reciclaje de paneles solares, tomando en cuenta la recuperación del vidrio. Primero, utilizó un método mecánico para fragmentar el módulo solar, separando las muestras por tamaño de partícula: mayores a 1mm, de 0.4 a 1mm, de 0.08 a 0.4mm y menores a 0.08mm. Las fracciones mayores a 1mm fueron llevadas a un tratamiento químico con ciclohexano, 0.2g por 1ml de solvente a 60°C durante una hora con agitación. Las fracciones de 0.08 a 0.4mm fueron llevadas a un proceso de mineralización, 1g de esta fue puesta en 3ml de ácido nitrico, 9 ml de ácido clorhidrico y 0.6 ml de peróxido de hidrógeno, con una rampa de temperatura 1:10 min de temperatura ambiente a 220°C en un digestor de microondas a 1000W. Una vez alcanzados los 220$^{\circ}$C, la muestra se mantuvo durante 20 minutos a esta temperatura. Como resultado, se logró recuperar de las fracciones mayores a 1mm el 18\% del peso inicial, con tamaños de 0.4 a 1mm el 63\%. El 16\% del vidrio con muestras de 0.08 a 4mm y 2.1\% de las menores a 0.08\%. Esto es debido a que el vidrio se encontró en las muestras mayores a 0.4mm, mientras que había mayor cantidad de metales en las muestras menores a 0.08mm. Sin embargo, para las muestras mayores a 1mm fue necesario utilizar un proceso térmico para poder eliminar los restos de encapsulante EVA del vidrio. Este tipo de procesos, aunque efectivos, implican una fragmentación significativa del módulo, lo cual podría considerarse una limitación frente a métodos que buscan mantener los componentes más íntegros.
 

\section{Planteamiento del problema}
\label{sec:Planteamiento}

Se han realizado avances en la investigación de reciclaje de paneles solares, los métodos dependen del material final que se desea obtener: la celda solar completa para su reutilización o el silicio para su reciclaje. Para obtener la celda completa se requieren procesos químicos que involucran solventes y altas temperaturas para disolver el encapsulante, que muchas veces presenta hinchazón causando daños físicos a celda, por lo que no se considera un método viable \citep{Kim2012}. Cuando se desea obtener el silicio de la celda, los solventes y altas temperaturas no representan un problema. Se han presentado varios artículos de obtención de silicio a partir del reciclaje de paneles solares \citep{Azeumo2019,Ardente2019,Fiandra2019}, sin embargo, estos no mencionan qué pasa una vez que obtuvieron el material final. Es necesario estudiar la calidad de silicio obtenido en el proceso de reciclaje para poder darle el uso correcto, ya sea para crear nuevas celdas solares (impurezas menores a 0.001 p.p.m) u otros dispositivos electrónicos (impurezas menores a 0.1 p.p.m)\citep{Martil2016}.  

Los principales componentes del panel solar son el vidrio (65–85\% en peso), el aluminio (10–20\% en peso) y el encapsulante (5–10\% en peso). Los demás elementos, como la celda y los metales conductores, representan menos del 4\% en peso del panel solar \citep{Fiandra2019}.. A pesar del alto porcentaje que tienen los otros componentes, los estudios de reciclaje están enfocados en la obtención de silicio y metales, debido a que son los materiales los cuales su reciclaje trae el mayor beneficio económico. Los incentivos económicos de los demás materiales son bajos por lo que los hace poco atractivos para el reciclaje \citep{SamKumar2020}.  


\section{Justificación}
\label{sec:Justificación}
Comparada con otras fuentes de energía, la solar fotovoltaica es una tecnología limpia, versátil y abundante que ha crecido significativamente durante la última década. Con el fin de limitar el impacto negativo del crecimiento de volumen de residuos fotovoltaicos, se ha propuesto el reciclaje de estos. Actualmente los procesos de reciclaje se enfocan en la obtención de silicio y metales como la plata debido al beneficio económico. La tecnología de paneles fotovoltaicos de silicio sigue evolucionando y se espera que al 2030, el porcentaje de peso de los metales se reduzca en 0.01\%. Actualmente el uso de plata es cada vez menor, ya que mediante mejores procesos de metalización está siendo reemplazada por cobre o níquel/cobre. A su vez, se espera que el porcentaje silicio utilizado en el módulo solar baje de 5 a 3\%. Contrario a esto, se estima que el vidrio represente más del 80\% del peso del panel fotovoltaico (IRENA e IEA, 2016). Estos cambios reflejan una tendencia hacia módulos más livianos en metales y más sostenibles. sin embargo, también generan nuevos retos para los procesos de reciclaje. La directiva de WEEE de la UE ha establecido que el reciclaje de alto valor también debe incluir la obtención de elementos de alto valor energético, como el silicio y vidrio. A su vez, deben considerar la calidad del material recuperado y la eliminación de sustancias potencialmente dañinas\citep{Pagnanelli2019,Irena2016}. 
Es importante evaluar los métodos, para así poder disminuir el impacto negativo de los módulos fotovoltaicos en el ambiente. Es necesaria una estrategia que incluya el reciclaje y/o correcto manejo de todos los componentes, tomando en cuenta el manejo de residuos de los procesos involucrados.


\section{Objetivo general}
\label{sec:Objetivo general}

Evaluar métodos de reciclaje de paneles solares para la obtención de materiales de alto valor agregado tomando en cuenta el impacto ambiental del proceso. 

\section{Objetivos específicos}
\label{sec:Objetivos específicos}
1.- Evaluar métodos separar vidrio y backsheet. 

2.- Evaluar métodos para la separación del EVA de la celda.

3.- Valorar la recuperación de silicio y otros metales.

4.- Analizar la toxicidad de los residuos del proceso. 



