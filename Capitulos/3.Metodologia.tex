Se trabaja con paneles solares monocristalinos (Yingli, modelo YL260C) y policristalinos(Perlight, modelo PLM260). A los módulos se les retira previamente el marco de aluminio y la caja de conexiones. Se cortan en muestras de 10x10 cm compuestas de backsheet, vidrio, celda y encapsulante EVA, estas serán nombradas como M-inicial en este escrito. 

Para la obtención del backsheet, se utilizará un método mecánico, sin embargo para los demás componentes se evaluarán dos métodos: químico y mecánico para la separación del vidrio; químico y térmico para la obtención de celda solar y EVA. 

Se realizará una evaluación de los residuos generados además de propuestas de disposición y reciclaje. El esquema general del proceso se muestra en figura \ref{fig:EsquemaGeneral}.   

\begin{figure}[htb]
	\begin{center}
		\includegraphics[width=1\textwidth]{./Figuras/diagramarutas.png}
	\end{center}
	\vspace{-1em} %Para reducir el espacio antes del caption
	\caption{Esquema general de evaluación de métodos para obtención de componentes de M-inicial}
	\label{fig:EsquemaGeneral}
\end{figure}

\section{Recuperación de Backsheet}
\label{sec:Backsheet}

\subsection{Método mecánico}
\label{sec:Metodo mecánico}


 El proceso consiste en calentar la muestra con una pistola de calor, para poder retirar la placa haciendo "peeling". 


\textbf{Materiales}\\ 
• Pistola de calor\\ 
• Pinzas\\  
• Guantes\\ 

\textbf{Procedimiento:}\\ 
1.	La M-inicial se coloca con el backsheet hacía arriba y se calienta con la pistola de calor.\\ 
2. Después de unos minutos, con ayuda de las pinzas, se sostiene la muestra y se retira el backsheet cuidadosamente. \\

\section{Recuperación de vidrio}
\label{sec:Vidrio}

Después de retirar el backsheet de la M-inicial, esta queda compuesta por vidrio y la celda solar encapsulada con EVA, a la que llamaremos M-VEC. Se evalúan dos métodos para retirar el vidrio: químico y mecánico

\subsection{Método químico para separar vidrio}
\label{sec:Metodo químico para separar vidrio}

\textbf{Materiales}\\
• Vaso precipitado
• Tolueno\\
• Xileno\\
• Parrilla térmica\\
• Pastilla agitadora\\

\textbf{Procedimiento:}\\ 
1.	En el vaso precipitado de introducen 5 gr de M-VEC. 
2.- Se agregan 80 ml de solvente 
3.- Al vaso con solvente y muestra se lleve a agitación en la parrilla. 
4.- Se decantan los trozos de vidrio separados y el material precipitado es llevado a filtración recolectando el solvente. 

\subsection{Método mecánico para separar vidrio}
\label{subsec:Metodo mecánico para separar vidrio}

\textbf{Materiales:}\\ 
• Pistola de calor\\ 
• Cuchilla o cuter 
• Guantes\\
• Sujetadores (sargentos)\\

\textbf{Procedimiento:}\\ 
1. La M-VEC se coloca con el vidrio hacía arriba y se sujeta con los sargentos.\\ 
2. Se calienta el vidrio con la pistola de calor por 3 min con velocidad de aire baja y temperatura alrededor de 70 °C
2. Sin detener el aire caliente, se pasa la cuchilla entre el vidrio y el EVA, desprendiéndolo. 



\section{Recuperación de celda solar}
\label{sec:Recuperación de celda solar} 

Del proceso de retiro de vidrio, obtuvimos dos distintas muestras: las obtenidas químicamente y las que se recuperaron mecánicamente. Con el fin de distinguir unas de otras, llamaremos CEQ a las obtenidas químicamente y CEM a las obtenidas mecánicamente. 
Se evalúan dos proceses para la recuperación de silicio y metales de la celda solar.  

\subsection{Proceso químico}
\label{sec:proceso químico}

\textbf{Materiales:}\\ 
• Baño maría\\  
• Espátula\\ 
• Recipiente\\ 
• Tolueno\\
• Agua destilada\\
• Centrifugadora\\  

\textbf{Procedimiento para muestras CEQ:}\\ 
1.	Se coloca 0.5g de muestra en el recipiente con 5 ml de solvente\\ 
2. Una vez bien cerrado el recipiente se coloca en una rejilla para ser llevada a baño maría a 80°C durante 120 min.\\
3. Después del tiempo en baño maría, las muestras se llevan a centrifugación durante 30 min a 3600 RPM.\\
4. Se decanta el EVA de la superficie y se recolecta los trozos de celda precipitada.\\
5. El solvente restante se analiza mediante cromatografía de gases para identificar los subproductos extraídos del EVA. Esta información permite evaluar el posible impacto ambiental y la necesidad de tratamiento del solvente antes de su reutilización o disposición.\\
6.- El EVA decantado es llevado a FTIR para su análisis.

\textbf{Procedimiento para muestras CEM:}\\
Las CEM son diferentes a la CEQ, ya que estas se recuperan como una placa de 10x10 cm de EVA encapsulando la celda, el proceso químico es el siguiente:

1. Los conectores de la muestra se pueden retirar con la ayuda de pinzas, una vez retirados, la CEM es cortada en trozos de aproximadamente 2x2 cm.
2. Se tritura la muestra con una molino de granos de acero\\
3. La CEM obtenida es triturada es tamizada por una malla No. 16.\\
4. Se lleva la muestra tamizada a baño maría con tolueno a 80°C por 1 hr. Mientras que el EVA sobre la malla se lava con Xileno a temperatura ambiente por 1hr con agitación.\\
5. Después del tiempo en baño maría, las muestras se llevan a centrifugación durante 30 min a 3600 RPM.\\
6. Se decanta el EVA de la superficie y se recolecta los trozos de celda precipitada.\\
7.El solvente restante se analiza mediante cromatografía de gases para identificar los subproductos extraídos del EVA. Esta información permite evaluar el posible impacto ambiental y la necesidad de tratamiento del solvente antes de su reutilización o disposición.\\
8.- El EVA lavado se recolecta con el EVA decantada y se analiza por FTIR.


\subsection{Proceso térmico}
\label{sec:proceso térmico}

Para ambas muestras, CEP y CEM, el proceso térmico es el que se describe a continuación.

Materiales:\\ 
• Horno cilíndrico\\ 
• Crisol\\ 
• Espatula\\ 
• Guantes\\ 
• Bolsa recolectora de gases\\
• Cromatográfo de gases\\


Procedimiento:\\ 
1.	Se coloca la muestra en el crisol\\
2.  Se programa el horno de acuerdo a la muestra a ingresar \\
3.  Se posiciona el crisol en el horno y se cierra\\
4.  Los gases emitidos del proceso se recolectan y se llevan a cromatografía de gases\\
4.  Después de 1 hora se deja enfriar y se retira del horno 

El proceso térmico se avaluará en muestras de distinta composición con el fin de analizar la diferencia entre los gases emitidos. Se establecen las rampas del horno de acuerdo a la temperatura para degradar el polímero: 800°C para el backsheet y 500°C para el EVA. Se propone la configuración: 

\begin{itemize}
	\item \textbf{Backcheet/EVA/Celda/Vidrio}: 24-250°C 30m min/ 250-550°C 20 min/ 550-850°C 20 min / 800°C-1hr 
	\item \textbf{EVA/Celda/Vidrio}: 24-250°C 30m min/ 250-550°C 20 min/ 550°C-1hr 
	\item \textbf{EVA/Celda}: 24-250°C 30m min/ 250-550°C 20 min/ 550°C-1hr
	\item \textbf{EVA/Celda}: 250-550°C 50m min/ 550°C-1hr
	
\end{itemize}